\documentclass[
        oneside,      %%coloque  % no in\'icio desta linha para imprimir frente e verso 
        english			
%	french,				
%	spanish, 
%        portuguese			 
        ]{abntbibufjf}


\usepackage[T1]{fontenc}		
\usepackage[utf8]{inputenc}		%% Para converter automaticamente acentos como digitados normalmente no teclado. Mude utf8 para latin1 se precisar. 

\usepackage{lmodern} %no caso do modelo Latex, pode-se usar a fam\'ilia de fontes lmodern como aqui indicado, no lugar de Arial e Times New Roman.


\usepackage{lastpage}			
\usepackage{indentfirst}		
\usepackage{color}			
\usepackage{graphicx}			
\usepackage{microtype} 
\usepackage{hyperref}
\usepackage{xurl}
\usepackage{amsmath}
\usepackage{amssymb}

%% -----------------------------------------------------------------------------

%% Obs.: Alguns acentos foram omitidos.

\titulo{Desenvolvimento de um Analisador de Espectro Óptico de Baixo Custo para Faixa Visível} %% Colocar, dentro de chaves {}, o t\'itulo do trabalho. 
\subtitulo{Calibração Automatizada via Visão Computacional}  %% Colocar % no in\'icio desta linha se nao tiver subt\'itulo 
\autor{Jakson Almeida} %%Colocar, dentro de chaves {}, o nome completo do autor
\autorVirg{Almeida, Jakson} %%Colocar o sobrenome do autor, separado por v\'rgula, antes do restante do nome do autor. Ex.: Santos, Maria dos
\local{Juiz de Fora} %%Governador Valadares % N\~ao usar MG.
\data{2025} %%Colocar o ano da entrega. Por exemplo, 2019. N\~ao usar m\^es.
\orientador[Orientador:]{Alexandre Bessa dos Santos} %%Se precisar, troque [Orientador:] por [Orientadora:]
\coorientador[Coorientador:]{Felipe Barino} %% Colocar ``%'' no in\'icio desta linha se n\~ao tiver coorientador. Se precisar, troque por [Cooorientadora:]. 
\orientadorTitulo{Professor} %%Colocar, dentro de chaves {}, a titula\c{c}\~ao do(a) orientador(a). Por exemplo, Prof. Dr.
\coorientadorTitulo{Aluno de doutorado} %%Colocar, dentro de chaves {}, a titula\c{c}\~ao do(a) cooorientador(a). 
\instituicao{Universidade Federal de Juiz de Fora}
\faculdade{Faculdade de Engenharia El\'etrica} %%Colocar, dentro de chaves {}, o nome da faculdade ou do instituto.
\programa{Engenharia El\'etrica} %%Colocar, dentro de chaves {}, o nome do curso. Por exemplo: Programa de P\'os\mbox{-Gra}dua\c{c}\~ao em Matem\'atica
\objeto{Trabalho de Conclus\~ao de Curso (gradua\c{c}\~ao)}  %%Tese (Doutorado)  %%%Trabalho de Conclus\~ao de Curso (gradua\c{c}\~ao)
\natureza{Trabalho de conclus\~ao de curso apresentado \'a \inserefaculdade da   %% %%%Trabalho de conclus\~ao de curso apresentado \'a \inserefaculdade da %%%%SUBSTITUIR \'a POR ao SE FOR INSTITUTO    
Universidade Federal de Juiz de Fora como requisito parcial \`a obten\c{c}\~ao do 
grau de bacharel em  %%Doutor em    %%%grau de bacharel em 
Engenharia Elétrica. %%Trocar Matem\'atica por outro, se precisar.
}

%% Abaixo, prencher com os dados da parte final da ficha catalografica
\finalcatalog{1. Analisador de espectro óptico. 2. Calibração automatizada. 3. Visão computacional. 4. Espectro visível. I. Sobrenome, Nome do orientador, orient. II. T\'itulo.} %% Aqui fica 
% escrito a palavra ``T\'itulo'' mesmo, nao o do trabalho. Se tiver coorientador, os dados ficam depois dos dados 
%% do orientador (II. Sobrenome, Nome do coorientador, coorient.) e antes de ``II. T\'itulo'', o qual passa a ``III. T\'itulo''.


\usepackage[round, numbers]{natbib} %para refer\^encias bibliogr\'aficas no sistema num\'erico com () na chamada da citacao. 

%Se for usar o sistema autor-data, colocar % antes de \usepackage acima e retirar % antes de \usepackage abaixo.

%\usepackage{natbib} %para o sistema autor-data

% Remove negrito do título na capa e folha de rosto
\makeatletter
\renewcommand{\inserecapa}{
  \begin{capa}
   \begin{center}
    {
    \bfseries \MakeUppercase{\insereinstituicao} \par \MakeUppercase{\inserefaculdade} \par \MakeUppercase{\insereprograma}}
    \vfill
    {
    \bfseries \insereautor}
    \vfill
    {\normalfont \inseretitulo \IfNonempty{\inseresubtitulo}{: \inseresubtitulo}}
    \vfill
    \inserelocal \par \inseredata 
   \end{center}
   \end{capa}
}

\renewcommand{\folhaderostocont}{
  \begin{center}
    {
    \bfseries \insereautor} 
    \vspace*{\fill}
    \begin{center}
      {
      \normalfont \inseretitulo 
        \IfNonempty{\inseresubtitulo}{: \inseresubtitulo}}
    \end{center}
    \vspace*{\fill}
      \IfNonempty{\inserenatureza}{
      \hspace{.45\textwidth}
      \begin{minipage}{.5\textwidth}
      	\SingleSpacing
         \inserenatureza
      \end{minipage}
       \vspace*{\fill}
    }
  \end{center} 
   {\insereorientadorOU ~\insereorientadorTitulo ~\insereorientador \par }
    \IfNonempty{\inserecoorientador}{
       {\inserecoorientadorOU ~\inserecoorientadorTitulo ~\inserecoorientador}
    }
    \vfill
  \begin{center}  
    \inserelocal \par \inseredata   
  \end{center}
}
\makeatother

\begin{document}

%% ELEMENTOS PR\'E-TEXTUAIS


%% Capa. N\~ao entra na contagem da pagina\c{c}\~ao.
\inserecapa

%% Folha de rosto
\inserefolhaderosto

%% Ficha catalogr\'afica. AO IMPRIMIR, DEIXAR NO VERSO DA FOLHA DE ROSTO.
\inserecatalog  


%% Folha de aprovacao. Incluir mesmo que sem assinaturas. Assinaturas eletr\^onicas via SEI.
\begin{folhadeaprovacao}
\inicfolhaaprov
        
Aprovada em (dia) de (m\^es) de (ano) %%Preencher com a data 
   
\vfill
\begin{center} BANCA EXAMINADORA \end{center}
\assinatura{\insereorientadorTitulo~\insereorientador \ - Orientador \\ Universidade Federal de Juiz de Fora}  %%Orientadora
%\assinatura{Professor Dr. \inserecoorientador \ - Coorientador \\ Universidade Federal de Juiz de Fora}
\assinatura{Aluno de Doutorado Felipe Barino - Coorientador \\ Universidade Federal de Juiz de Fora}
\assinatura{Professor André Luiz Marques Marcato \\ Universidade Federal de Juiz de Fora} 
%\assinatura{...} %%RETIRE O % E PREENCHA SE PRECISAR
%  \assinatura{...}
%  \assinatura{...}
\end{folhadeaprovacao}
\cleardoublepage 


%% Dedicatoria. OPCIONAL. N\~ao deve haver t\'itulo. Colocar ``%'' no in\'icio de cada das 3 linhas abaixo, caso n\~ao queira. 
 \begin{dedicatoria} 
  Dedico este trabalho à minha família, que me apoiou em todos os passos desta jornada, e a todos que contribuíram para minha formação.
 \end{dedicatoria}

 
%% Agradecimentos. OPCIONAL. Caso seja bolsista, inserir os devidos agradecimentos.
\begin{agradecimentos}
Agrade\c{c}o aos meus pais e irm\~aos por todo o apoio durante esses anos de estudos longe de casa. Seu amor e incentivo foram fundamentais para que eu pudesse trilhar este caminho.

Agrade\c{c}o ao Professor Andr\'e Luiz Marques Marcato por ter me introduzido \`a Inicia\c{c}\~ao Cient\'ifica. Durante os quatro anos em que trabalhamos juntos, aprendi ROS, simula\c{c}\~ao de drones e vis\~ao computacional --- \'areas diretamente relacionadas \`a minha forma\c{c}\~ao em Engenharia El\'etrica com habilita\c{c}\~ao em Rob\'otica e Automa\c{c}\~ao Industrial.

Agrade\c{c}o em especial ao Professor Alexandre Bessa dos Santos, orientador deste trabalho, com quem atuei nos \'ultimos dois anos na \'area de instrumenta\c{c}\~ao \'optica. O presente trabalho d\'a continuidade a uma Inicia\c{c}\~ao Cient\'ifica desenvolvida em seu laborat\'orio.

Agrade\c{c}o ao coorientador Felipe Barino, que tamb\'em integra o LITel e me apresentou ao projeto OSA Visível que deu origem a este TCC, compartilhando seu conhecimento e dedicando tempo para me acompanhar neste percurso.

Por fim, agrade\c{c}o a tantos outros professores e colegas que, de diversas formas, contribu\'iram para a minha forma\c{c}\~ao acad\^emica. S\~ao muitas pessoas que merecem meu reconhecimento; opto por n\~ao nominar individualmente para n\~ao correr o risco de omitir algu\'em. A todos, meu sincero agradecimento.
\end{agradecimentos}


%% Ep\'igrafe. OPCIONAL. Com os dados do autor. A obra usada na ep\'igrafe deve constar nas refer\^encias. 

% Quando at\'e 3 linhas: \'e obrigat\'orio o uso de aspas duplas.

%\begin{epigrafemenos} %%Ep\'igrafe com 3 ou menos linhas
%``Mas para que o produto de uma pesquisa científica possa ser publicado não basta que ele apresente um conteúdo de qualidade, também é exigida qualidade de forma.'' (MAR\c{C}AL JUNIOR, 2013, p. 19-20).
%\end{epigrafemenos}

%%Quando com mais de 3 linhas. 

\begin{epigrafemais} %%Ep\'igrafe com mais de 3 linhas 
	Elemento opcional, em que o autor apresenta uma cita\c{c}\~ao, seguida de indica\c{c}\~ao de autoria, relacionada com a                       
  mat\'eria tratada no corpo do trabalho. (Associa\c{c}\~ao Brasileira de Normas T\'ecnicas, 2011, p. 2).
\end{epigrafemais}


%% RESUMOS

%% Resumo em Portugu\^es. OBRIGAT\'ORIO. \'E obrigat\'orio o uso de par\'agrafo \'unico.
\begin{resumo}

Este trabalho apresenta o desenvolvimento de um Analisador de Espectro Óptico (OSA) de baixo custo para a faixa visível (380--750 nm), combinando um espectrômetro impresso em 3D ("Osinha") e um software customizado ("OSA Visível"). O processo de calibração emprega técnicas de visão computacional para mapear a relação comprimento de onda-posição de pixel, utilizando luz branca e lasers (532 nm e 650 nm). Etapas-chave incluem cálculo de centróide, limiarização de intensidade e regressão linear via bibliotecas OpenCV e Pillow. O software atinge precisão de $\pm$1.8 nm, validada em experimentos com fontes de luz controladas. Ao reduzir custos de mais de \$30.000 (OSA Comercial) para menos de \$200, esta solução democratiza a análise espectral para aplicações educacionais e de pesquisa. A interface intuitiva permite visualização em tempo real e exportação de dados, dispensando conhecimentos em programação. \\[18pt]
Palavras-chave: analisador de espectro óptico; calibração automatizada; visão computacional; espectro visível; regressão linear. %Separadas por ``;'' e iniciadas por letras min\'usculas.
\end{resumo}
 
 
%% Resumo em Ingl\^es. \'E obrigat\'orio o uso de par\'agrafo \'unico.
\begin{resumo}[ABSTRACT]
 \begin{otherlanguage*}{english}

This work presents the development of a low-cost Optical Spectrum Analyzer (OSA) for visible light spectra (380--750 nm), combining a 3D-printed spectrometer ("Osinha") and custom software ("OSA Visível"). The calibration process employs computer vision techniques to map wavelength-to-pixel relationships using white light and laser sources (532 nm and 650 nm). Key steps include centroid calculation, intensity thresholding, and linear regression via OpenCV and Pillow libraries. The software achieves a calibration accuracy of $\pm$1.8 nm, validated through experiments with controlled light sources. By reducing costs from over \$30,000 (commercial OSA) to under \$200, this solution democratizes spectral analysis for educational and research applications. The intuitive interface allows real-time visualization and data export, eliminating the need for specialized programming skills. \\[18pt]
Keywords: optical spectrum analyzer; automated calibration; computer vision; visible spectrum; linear regression. %Separadas por ``;'' e iniciadas por letras min\'usculas.
 \end{otherlanguage*}
\end{resumo}

%% Seguindo o mesmo modelo acima, pode-se inserir resumos em outras l\'inguas. 



%% Lista de ilustra\c{c}\~oes. OPCIONAL. Sao consideradas ilustra\c{c}\~oes: desenhos, esquemas, fluxogramas, figuras, fotografias, gr\'aficos, mapas, organogramas, plantas, quadros, entre outros. Tabelas n\~ao s\~ao consideradas ilustra\c{c}\~oes. A ordem da lista deve obrigatoriamente ser a mesma ordem em que as ilustra\c{c}\~oes aparecem no texto. Quando o t\'itulo ocupar mais de uma linha, a segunda linha deve estar exatamente abaixo da primeira.  

\pdfbookmark[0]{\listfigurename}{lof}

%Caso as ilustra\c{c}~oes do trabalho sejam todas do mesmo tipo (por exemplo, todas do tipo organograma), coloque % no in\'icio das duas linhas abaixo. 
\ilustvaria   %Use este comando somente caso as ilustra\c{c}\~oes n\~ao sejam todas do mesmo tipo. 
\listilustvaria  %Use este comando somente caso as ilustra\c{c}\~oes n\~ao sejam todas do mesmo tipo e caso queira inserir a lista delas. 

%\listoffigures*  %Use este comando quando todas as ilustra\c{c}\~oes s\~ao do mesmo tipo e caso queira inserir a lista delas. Veja dicas no final deste arquivo.

\cleardoublepage
\pdfbookmark[0]{\listtablename}{lot}

%% Lista de tabelas. OPCIONAL. A ordem da lista de tabelas deve obrigatoriamente ser a mesma ordem em que as tabelas aparecem no texto. 


\listoftables*    %Coloque ``%'' no in\'icio desta linha, caso n\~ao queira lista de tabelas. 

\cleardoublepage


%% Lista de abreviaturas e siglas. OPCIONAL. Nao deve haver sinal grafico entre as siglas e abreviaturas e o significado. 

\begin{siglas} %%ALTERAR OS EXEMPLOS ABAIXO, CONFORME A NECESSIDADE
  \item[BBL] Beer-Bouguer-Lambert (Lei de Beer-Bouguer-Lambert)
  \item[CMOS] Complementary Metal-Oxide-Semiconductor (Semicondutor de Óxido Metálico Complementar)
  \item[COTS] Commercial Off-The-Shelf (Componentes Comerciais Prontos para Uso)
  \item[CRI] Color Rendering Index (Índice de Reprodução de Cor)
  \item[DWDM] Dense Wavelength Division Multiplexing (Multiplexação por Divisão de Comprimento de Onda Densa)
  \item[FBG] Fiber Bragg Grating (Rede de Bragg em Fibra)
  \item[HCN] Cianeto de Hidrogênio (Hydrogen Cyanide)
  \item[IEC] International Electrotechnical Commission (Comissão Eletrotécnica Internacional)
  \item[LITel] Laboratório de Instrumentação e Telemetria
  \item[LPG] Long-Period Grating (Rede de Longo Período)
  \item[OSA] Optical Spectrum Analyzer (Analisador de Espectro Óptico)
  \item[OSNR] Optical Signal-to-Noise Ratio (Relação Sinal-Ruído Óptico)
  \item[POC] Point-of-Care (Ponto de Cuidado)
  \item[RGB] Red, Green, Blue (Vermelho, Verde, Azul)
  \item[ROI] Region of Interest (Região de Interesse)
  \item[RMS] Root Mean Square (Raiz Quadrada da Média dos Quadrados)
  \item[SNR] Signal-to-Noise Ratio (Relação Sinal-Ruído)
  \item[UFJF] Universidade Federal de Juiz de Fora
  \item[USD] United States Dollar (Dólar Americano)
  \item[UV-Vis] Ultravioleta-Visível
  \item[VIS] Visível (faixa espectral)
  \end{siglas}

%% Lista de s\'imbolos. OPCIONAL. Nao deve haver sinal grafico entre o simbolo e o seu significado.

\begin{simbolos} %%ALTERAR OS EXEMPLOS ABAIXO, CONFORME A NECESSIDADE
  \item[$ \lambda $] Comprimento de onda
  \item[$ T $] Transmitância
  \item[$ I $] Intensidade da radiação transmitida
  \item[$ I_0 $] Intensidade da radiação incidente
  \item[$ A $] Absorbância
  \item[$ \epsilon $] Absortividade molar
  \item[$ c $] Concentração da substância em solução
  \item[$ l $] Comprimento do caminho óptico
  \item[$ R^2 $] Coeficiente de determinação
  \item[$ n $] Ordem de difração
  \item[$ d $] Espaçamento da grade de difração
  \item[$ \theta $] Ângulo de difração
  \item[$ \alpha $] Ângulo de incidência
  \item[$ a, b $] Coeficientes da regressão linear $\lambda(x) = a \cdot x + b$
  \item[$ I(x,y) $] Intensidade do pixel na posição (x,y)
  \item[$ M(x,y) $] Máscara binária após limiarização
  \item[$ x_c, y_c $] Coordenadas do centróide
  \item[$ x_{peak} $] Posição do pico de intensidade
 \end{simbolos}

 
%% Sum\'ario

\pdfbookmark[0]{\contentsname}{toc}
\tableofcontents*
\cleardoublepage

%% ----------------------------------------------------------

%% ELEMENTOS TEXTUAIS
\textual


\chapter{INTRODU\c{C}\~AO}  %%Nesta linha, dentro de { }, digita-se em CAIXA ALTA, como apresentado aqui

O Analisador de Espectro Óptico (OSA) é um instrumento metrológico cuja importância reside na sua capacidade de quantificar a distribuição de potência da radiação eletromagnética em função do comprimento de onda ($\lambda$). Tradicionalmente, o campo de estudo da espectrofotometria Ultravioleta-Visível (UV-Vis) abrange a região que vai de aproximadamente $190$ nm a $900$ nm \cite{agilent_uvvis}. Dentro deste intervalo, a faixa visível (VIS) é estritamente definida, compreendendo comprimentos de onda entre aproximadamente $400$ nm (violeta) e $700$ nm (vermelho).

Esta região espectral é particularmente crítica, pois as transições de energia atômicas e moleculares que ocorrem neste domínio são a base de diversas aplicações químicas, biológicas e de metrologia de cor. A radiação UV e VIS possui energia suficiente para induzir transições entre diferentes níveis de energia eletrônica em moléculas \cite{agilent_uvvis}. Quando a radiação eletromagnética interage com a matéria, diversos fenômenos podem ocorrer, incluindo reflexão, espalhamento, absorção, fluorescência/fosforescência e reações fotoquímicas \cite{agilent_uvvis}.

A relevância do OSA na faixa visível está intrinsecamente ligada à capacidade de quantificar a presença de compostos ou biomarcadores que exibem absorção seletiva nesta faixa. Por exemplo, a utilização da faixa visível (em especial o vermelho, $660$ nm) para diagnósticos, como na oximetria de pulso \cite{smartphone_spectroscopy}, não é acidental, mas sim uma consequência direta das diferenças de absorção espectral entre a hemoglobina oxigenada e a desoxigenada. Essa sensibilidade à composição química enfatiza a necessidade fundamental de calibração ultra-precisa do eixo de comprimento de onda ($\lambda$). Um desvio mínimo no mapeamento espectral pode resultar em erros significativos na medição da concentração, particularmente em regiões de alta inclinação do espectro de absorção.

Um Analisador de Espectro Óptico (OSA) é um instrumento de precisão projetado para medir e exibir a distribuição de \textbf{potência óptica} de uma fonte de luz sobre um determinado intervalo de comprimento de onda \cite{viavi_osa}. Os três parâmetros-chave quantificados por um OSA são o comprimento de onda, o nível de potência e a Relação Sinal-Ruído Óptico (\textit{Optical Signal-to-Noise Ratio} - OSNR) \cite{viavi_osa}. É crucial distinguir a funcionalidade de um OSA de um espectrofotômetro tradicional de bancada. O termo OSA é frequentemente empregado para medir a distribuição espectral de potência de uma fonte de luz (Potência vs. $\lambda$) \cite{viavi_osa}, sendo amplamente utilizado em telecomunicações para caracterizar lasers. No contexto da caracterização de sensores de fibra óptica, como \textit{Long-Period Gratings} (LPGs) e \textit{Fiber Bragg Gratings} (FBGs), o OSA atua como um \textbf{interrogador espectral}, medindo a atenuação ou o pico de reflexão \cite{fiber_sensor_characterization}.

Analisadores de Espectro Óptico comerciais que operam na faixa visível apresentam custos elevados, frequentemente superiores a \$30.000, conforme pesquisa realizada em março de 2025 \cite{thorlabs_osa20xc}. Isso restringe significativamente o acesso desses equipamentos para muitas instituições acadêmicas e laboratórios de pequeno porte, limitando o desenvolvimento de pesquisas e o ensino prático de espectroscopia. A necessidade de precisão na faixa visível é crítica, especialmente em medições ratiométricas como a oximetria de pulso, onde um erro mínimo na calibração do comprimento de onda (eixo $\lambda$) pode levar a erros diagnósticos significativos \cite{smartphone_spectroscopy}.

Como alternativa, o desenvolvimento de soluções de baixo custo tem ganhado destaque na literatura científica. O projeto "Osinha" foi desenvolvido com uma abordagem inovadora, utilizando impressão 3D, uma webcam e uma grade de difração, totalizando menos de \$200 USD, tornando-o uma solução acessível para pesquisa e ensino.

Este trabalho propõe uma solução de baixo custo baseada em hardware aberto ("Osinha") e software customizado ("OSA Visível"), desenvolvido pelo Laboratório de Instrumentação e Telemetria (LITel/UFJF). O princípio fundamental é o mesmo dos OSAs tradicionais: a luz, após interagir com a amostra ou sensor, é dispersa (geralmente por uma grade de transmissão simples) e, em vez de atingir um detector de varredura ou um \textit{array} de fotodiodos especializado, é capturada pelo sensor CMOS \cite{smartphone_spectroscopy}. O sensor CMOS, que é um detector de intensidade, torna-se o principal \textit{array} de detecção espectral do sistema.

O diferencial deste trabalho reside no uso de técnicas de visão computacional para calibração automática, substituindo métodos manuais ou baseados em hardware dedicado. O desafio metrológico primário em sistemas de visão é a conversão precisa dos dados de imagem (coordenada de pixel) em dados espectrais (comprimento de onda, $\lambda$) \cite{smartphone_spectroscopy}. O sensor CMOS, embora otimizado para o VIS ($400$ nm a $700$ nm) devido ao filtro infravermelho \cite{smartphone_spectroscopy}, captura a informação através de uma matriz de filtros Bayer (RGB). Isso significa que a medição não é puramente espectral, mas sim uma resposta de cor que deve ser desmembrada. O processo de dispersão óptica mapeia $\lambda$ para uma posição física no sensor ($p$). A relação entre $p$ e $\lambda$ não é trivialmente linear e é altamente suscetível a desalinhamentos físicos da grade de difração, à temperatura e a não-uniformidades na montagem.

A visão computacional é o motor que realiza a \textbf{conversão de pixel-para-wavelength} \cite{smartphone_spectroscopy}. Esta etapa é o cerne da calibração, transformando uma imagem de difração (dados de pixel RGB) em um espectro quantitativo (Intensidade vs. $\lambda$). A calibração desenvolvida neste trabalho combina duas etapas principais: (i) captura do espectro contínuo de luz branca, com detecção do centróide e ajuste inicial de uma reta via regressão linear; e (ii) correlação direta entre picos de intensidade de lasers (532 nm e 650 nm) e suas posições na imagem, definindo a relação comprimento de onda-posição de pixel.

Ao contrário dos OSAs tradicionais, que garantem a precisão através de componentes ópticos e eletrônicos caros (monocromadores de alta qualidade, arquiteturas de feixe duplo) \cite{agilent_uvvis}, o método deste trabalho utiliza software (visão computacional) para compensar as deficiências inerentes ao hardware óptico de baixo custo, como desalinhamentos na montagem ou não-linearidades introduzidas por lentes simples. Essa abordagem representa uma \textbf{mudança de paradigma metrológico}, onde a garantia de precisão migra da engenharia óptica de \textit{hardware} para a engenharia algorítmica de \textit{software}. O algoritmo de calibração deve ser robusto o suficiente para realizar a \textbf{compensação de instabilidade da fonte de luz} e a \textbf{correção de ruído} via \textit{software}, replicando as funções que os sistemas ópticos de alto custo executam em hardware \cite{agilent_uvvis}.

O objetivo geral deste trabalho é desenvolver um Analisador de Espectro Óptico de baixo custo para a faixa visível (400-700 nm), integrando hardware acessível e software customizado com calibração automatizada via visão computacional. A calibração do instrumento desenvolvido deve obrigatoriamente garantir a precisão nas duas dimensões metrológicas: a escala de \textbf{Potência (Eixo Y)} e a escala de \textbf{Wavelength (Eixo X)} \cite{viavi_osa}, replicando a complexa funcionalidade de um OSA comercial.

Os objetivos específicos incluem:

\begin{itemize}
    \item Desenvolver um espectrômetro 3D-printed de baixo custo baseado no projeto "Osinha";
    \item Implementar software "OSA Visível" para análise espectral com interface intuitiva;
    \item Criar sistema de calibração automatizada utilizando técnicas de visão computacional que realize a conversão precisa de pixel-para-wavelength \cite{smartphone_spectroscopy};
    \item Validar a precisão do sistema através de experimentos com fontes de luz controladas, atingindo resolução metrológica necessária para aplicações críticas (e.g., $\Delta\lambda < 0.1$ nm para sensores de fibra óptica) \cite{fbg_sensor};
    \item Reduzir custos de mais de \$30.000 (OSA comercial) para menos de \$200;
    \item Alcançar precisão de calibração de $\pm$2 nm na faixa visível, garantindo que a resolução de comprimento de onda eficaz seja comparável à de um OSA comercial.
\end{itemize}

A metodologia adotada compreende o desenvolvimento de hardware baseado em impressão 3D, implementação de algoritmos de visão computacional em Python, calibração híbrida com luz branca e lasers de referência, e validação experimental com fontes de luz controladas. Os métodos de calibração por visão geralmente envolvem: (i) localização geométrica, utilizando algoritmos de processamento de imagem para identificar e isolar a linha de difração na imagem \cite{smartphone_spectroscopy}; (ii) mapeamento espectral, utilizando fontes de calibração de referência (lâmpadas de vapor ou LEDs de pico estreito com comprimentos de onda conhecidos) para criar uma função de mapeamento entre a coordenada do pixel e o comprimento de onda conhecido; e (iii) processamento de dados, convertendo digitalmente a imagem e extraindo a curva quantitativa de intensidade vs. comprimento de onda \cite{smartphone_spectroscopy}.

A importância dos OSAs transcende o laboratório de pesquisa, estendendo-se a setores de alta tecnologia, saúde e manufatura. Nas telecomunicações, o OSA monitora a precisão do comprimento de onda para garantir que os canais estejam corretamente separados em redes DWDM \cite{viavi_osa}. Na caracterização de sensores de fibra óptica, como FBGs e LPGs, o OSA permite quantificar a sensibilidade com precisões da ordem de $0.037$ nm/kPa \cite{fbg_sensor}. Em aplicações biomédicas, como a oximetria de pulso, a precisão na faixa visível (especialmente em $660$ nm) é crítica para diagnósticos precisos \cite{smartphone_spectroscopy}. O desenvolvimento de um método de calibração por visão computacional habilita diretamente a transição industrial para dispositivos \textit{Point-of-Care} (POC) em ambientes com recursos limitados.

Este trabalho está organizado da seguinte forma: o Capítulo 2 apresenta a fundamentação teórica sobre espectroscopia óptica e analisadores de espectro; o Capítulo 3 detalha a metodologia de desenvolvimento do hardware e software, incluindo os algoritmos de visão computacional; o Capítulo 4 descreve os resultados experimentais e validação do sistema, comparando com padrões metrológicos estabelecidos; e o Capítulo 5 apresenta as conclusões e trabalhos futuros, destacando as contribuições para a metrologia de baixo custo.

\chapter{FUNDAMENTAÇÃO TEÓRICA}

Este capítulo apresenta a fundamentação teórica necessária para o desenvolvimento do Analisador de Espectro Óptico de baixo custo. Inicialmente, são abordados os fundamentos da espectroscopia óptica, incluindo a interação luz-matéria e a Lei de Beer-Lambert. Em seguida, são apresentados os princípios de difração e dispersão, fundamentais para o funcionamento do espectrômetro. Posteriormente, são discutidos os Analisadores de Espectro Óptico, suas arquiteturas e métodos de calibração, com ênfase nas técnicas modernas e na calibração por visão computacional. Por fim, é apresentado o processo híbrido de calibração desenvolvido neste trabalho.

\section{FUNDAMENTOS DA ESPECTROSCOPIA ÓPTICA}

A espectroscopia óptica constitui o pilar fundamental para a caracterização da matéria, explorando a interação entre a radiação eletromagnética e as propriedades atômicas e moleculares de uma amostra. O Analisador de Espectro Óptico (OSA) de baixo custo, denominado "Osinha"/"OSA Visível", é projetado para operar na faixa de luz visível (VIS, 380-750 nm). Neste intervalo, a absorção de energia luminosa tipicamente desencadeia transições eletrônicas em moléculas ou íons, permitindo a identificação e quantificação de compostos químicos através de seus espectros característicos.

\subsection{\textbf{Interação Luz-Matéria: Transmissão e Absorção}}

Quando a luz incide sobre uma amostra, ela pode ser transmitida, refletida ou absorvida. Em espectrofotometria, a análise primária foca na absorção. A luz branca, composta por todos os comprimentos de onda do espectro visível, tem sua intensidade reduzida de forma seletiva quando atravessa uma substância, resultando na cor percebida. Por exemplo, uma solução que absorve intensamente a luz verde parecerá vermelha ao observador. O objetivo do OSA é medir essa absorção seletiva em função do comprimento de onda.

A transmitância ($T$) é definida como a razão entre a intensidade da radiação transmitida ($I$) e a intensidade da radiação incidente ($I_0$), frequentemente expressa como porcentagem:

\begin{equation}
T = \frac{I}{I_0} \quad \textrm{ou} \quad \%T = \frac{I}{I_0} \times 100
\label{eq:transmitancia}
\end{equation}

A absorbância ($A$) é o parâmetro preferido para a quantificação, pois se relaciona linearmente com a concentração e o caminho óptico:

\begin{equation}
A = -\log T
\label{eq:absorbancia}
\end{equation}

\subsection{\textbf{Lei de Beer-Lambert: Base Quantitativa da Espectrometria de Absorção}}

A Lei de Beer-Lambert (também conhecida como Lei de Beer-Bouguer-Lambert, ou BBL) é a relação empírica que governa a atenuação da intensidade da radiação ao passar por um meio homogêneo \cite{beer_lambert_wiki}. Ela estabelece uma relação direta e linear entre a absorbância de uma solução e a concentração da substância absorvente, bem como o caminho óptico percorrido pela luz \cite{beer_lambert_edinst}.

A lei é expressa matematicamente como:

\begin{equation}
A = \epsilon \cdot c \cdot l
\label{eq:beer_lambert}
\end{equation}

onde $A$ representa a absorbância (adimensional), $\epsilon$ é a absortividade molar (uma constante característica da substância em um dado comprimento de onda), $c$ é a concentração da substância em solução, e $l$ é o comprimento do caminho óptico.

Para que a Lei de Beer-Lambert seja válida, várias premissas devem ser satisfeitas \cite{beer_lambert_wiki}. Estas incluem a necessidade de radiação monocromática, a ausência de interações entre as moléculas do soluto (o que geralmente requer baixas concentrações) e um meio macroscopicamente homogêneo. A alta correlação ($R^2 \geq 0.93$) observada durante os experimentos de validação do OSA Visível com a diluição de glicerina confirma que o sistema de baixo custo, quando devidamente calibrado, opera com a precisão necessária para manter a linearidade exigida pelo Beer-Lambert, tornando-o funcionalmente viável para aplicações quantitativas na faixa visível.

\subsection{\textbf{O Princípio da Difração de Fraunhofer e Grades Ópticas}}

A separação dos diferentes comprimentos de onda que incidem sobre o detector do OSA é realizada por um elemento dispersor, neste caso, uma grade de difração. A difração de Fraunhofer (também chamada de difração de campo distante) descreve o padrão de difração que ocorre quando as ondas planas incidem sobre o objeto dispersor e o padrão é observado a uma distância suficientemente longa ou no plano focal de uma lente de imagem, que é o arranjo típico de um espectrômetro \cite{fraunhofer_wiki}.

A relação fundamental que governa a dispersão angular de uma grade de difração é dada pela Equação da Grade:

\begin{equation}
n\lambda = d(\sin\theta + \sin\alpha)
\label{eq:diffraction}
\end{equation}

Nesta equação, $n$ representa a ordem de difração (geralmente $n=1$ é utilizado), $\lambda$ é o comprimento de onda da luz, $d$ é a constante da grade (espaçamento entre as linhas, $d = 1\,\mu\textrm{m}$ para a grade de $1000$ linhas/mm usada no Osinha), $\alpha$ é o ângulo de incidência da luz na grade, e $\theta$ é o ângulo de difração.

A relação entre o comprimento de onda ($\lambda$) e a posição física do pixel ($x$) no detector, resultante dessa dispersão, é inerentemente não linear, pois envolve funções trigonométricas de ângulo. No entanto, em sistemas compactos de baixo custo, como o "Osinha", o detector captura apenas uma porção angular muito pequena do espectro. Esta limitação angular permite que o mapeamento seja modelado, em primeira aproximação, por uma \textbf{Regressão Linear} ($\lambda(x) = a \cdot x + b$) \cite{henriksen_calibration}. Reconhece-se que essa simplificação linear requer uma calibração rigorosa e absoluta, que é justamente fornecida pelos pontos de referência de lasers de comprimento de onda conhecido, como será detalhado nas seções subsequentes.

\section{ANALISADORES DE ESPECTRO ÓPTICO: ARQUITETURAS E METROLOGIA}

O Analisador de Espectro Óptico (OSA) é um instrumento essencial para medir a distribuição de potência óptica em função do comprimento de onda \cite{spectrum_analyzer_wiki}.

\subsection{\textbf{Definição e Arquitetura do OSA Visível}}

Historicamente, OSAs utilizavam arquiteturas baseadas em monocromadores sintonizáveis (varredura), onde um filtro óptico ajustável resolvia os componentes espectrais individualmente. Em contraste, o OSA Visível utiliza uma arquitetura baseada em \textit{array} de detectores (grade/CMOS), que captura todo o espectro simultaneamente, melhorando a velocidade de aquisição \cite{keysight_osa}.

O hardware do espectrômetro "Osinha" foi desenvolvido com base no projeto \textit{Open Fiber Spectrometer} \cite{gaudi_spectrometer}, um espectrômetro de código aberto que fornece modelos 3D e instruções detalhadas para construção. O projeto original foi adaptado para atender às necessidades específicas deste trabalho, com foco na faixa visível do espectro (380-750 nm). O uso da impressão 3D reduz drasticamente os custos e permite a personalização rápida do design, superando a necessidade de polimento de precisão laboriosa, comum em óptica tradicional \cite{stanford_3d_optics}.

Os componentes principais são:

\begin{itemize}
    \item \textbf{Estrutura Impressa em 3D:} Proporciona a geometria necessária para o caminho óptico, fenda, e posicionamento da grade e do sensor.
    \item \textbf{Grade de Difração:} Responsável por separar os comprimentos de onda, com $1000$ linhas/mm.
    \item \textbf{Detector:} Uma Webcam USB, utilizando um sensor CMOS de $640 \times 480$ pixels.
\end{itemize}

O sinal de luz, depois de disperso, é discretizado pelo sensor CMOS. A intensidade digital no pixel $(x,y)$ é representada pela equação de amostragem:

\begin{equation}
I(x,y) = \sum_{k=0}^{255} k \cdot P(k|x,y)
\label{eq:sampling}
\end{equation}

onde $k$ é o nível de intensidade (de 0 a 255) e $P(k|x,y)$ é a probabilidade do pixel $(x,y)$ registrar a intensidade $k$.

\subsection{\textbf{Metrologia de OSAs: Padrões e a Justificativa para a Solução de Baixo Custo}}

A calibração de OSAs de alta precisão é regida por normas internacionais, como a IEC 62129, que detalham procedimentos para garantir a rastreabilidade da medição de comprimento de onda e potência \cite{iec_62129}. Métodos tradicionais de calibração utilizados em equipamentos de laboratório envolvem:

\begin{enumerate}
    \item \textbf{Fontes de Referência Estáveis:} Uso de lasers altamente estabilizados, cujos comprimentos de onda são monitorados simultaneamente por um \textit{wavemeter} de altíssima acurácia. \cite{dubard1995} discutiram a necessidade de caracterizar com precisão as fontes ópticas em medições de fibras ópticas e apresentaram técnicas de calibração para OSAs \cite{terra_calibration}.
    \item \textbf{Células de Gás:} Utilização de células de referência (e.g., HCN) que fornecem picos de absorção fixos e bem conhecidos em faixas específicas \cite{terra_calibration}.
    \item \textbf{Modelagem Polinomial Avançada:} Empregando lâmpadas de arco (neon, argônio) para gerar múltiplos picos espectrais conhecidos, ajustados por polinômios de segunda ou terceira ordem para mapear a relação comprimento de onda-pixel. \cite{liu2013} empregam métodos de calibração precisos utilizando parâmetros do sistema para espectrômetros de grade \cite{henriksen_calibration}.
\end{enumerate}

O custo desses métodos e dos equipamentos comerciais é proibitivo, variando de mais de \$30.000 a mais de \$100.000 para modelos de alta performance. Diante dessa realidade, o projeto "OSA Visível" representa uma abordagem estratégica para democratizar a análise espectral. A substituição da instrumentação metrológica de ponta pela \textbf{inteligência algorítmica} (visão computacional) permite alcançar uma acurácia funcional ($\pm 1.8$ nm) a um custo total inferior a \$200, justificando a inovação do processo de calibração automatizada.

A Tabela \ref{tab:comparativo_osa} apresenta um comparativo entre OSAs comerciais e o OSA Visível desenvolvido neste trabalho.

\begin{table}[ht]
\centering
\caption{Comparativo de Requisitos Metrológicos: OSA Comercial vs. OSA Visível}
\label{tab:comparativo_osa}
\begin{tabular}{>{\raggedright\arraybackslash}p{3cm}>{\centering\arraybackslash}p{2cm}>{\centering\arraybackslash}p{2.5cm}>{\centering\arraybackslash}p{2cm}>{\raggedright\arraybackslash}p{3cm}}
\hline
Modelo (Referência) & Faixa Espectral (nm) & Custo Estimado (USD) & Acurácia Típica & Método de Calibração \\
\hline
OSA201C \cite{thorlabs_osa20xc} & 350-1100 & > \$30.000 & Sub-nm & Interno, Rastreado a Padrões \\
AQ6380 \cite{yokogawa_aq6374} & 350-1750 & \$100.000-200.000 & Sub-pm & Padrões de Gás/Wavemeter \cite{terra_calibration} \\
\textbf{OSA Visível (Osinha)} & \textbf{380-750} & \textbf{< \$200} & \textbf{$\pm 1.8$ nm (RMS)} & \textbf{Híbrido (Visão Computacional + Laser)} \\
\hline
\end{tabular}
\fonte{Elaborada pelo autor (2025).}
\end{table}

\section{FUNDAMENTAÇÃO DO MAPEAMENTO COMPRIMENTO DE ONDA-PIXEL}

A calibração de comprimento de onda, ou calibração espectral, é o processo crucial que estabelece a correspondência exata entre o índice de pixel no sensor CMOS (coordenada $x$) e o valor absoluto do comprimento de onda ($\lambda$).

\subsection{\textbf{O Desafio da Relação $\lambda(x)$: Dispersão Física e Distorções do Sistema}}

A transformação do ângulo de difração ($\theta$) em posição de pixel ($x$) é influenciada pela geometria do sistema e pelas imperfeições do hardware. Em espectrômetros construídos com peças COTS (\textit{Commercial Off-The-Shelf}) e impressão 3D, existem fontes de erro que desviam a relação $\lambda(x)$ do ideal, como desalinhamentos angulares da grade, curvatura de campo do sensor e tolerâncias mecânicas na montagem.

\subsection{\textbf{A Modelagem de Primeira Ordem: Regressão Linear ($\lambda(x) = a \cdot x + b$)}}

Para simplificar o processamento e, dado que o OSA Visível opera em uma faixa espectral relativamente estreita (380-750 nm), o mapeamento é modelado com um ajuste de primeira ordem \cite{henriksen_calibration}. A relação linear entre o pixel e o comprimento de onda é definida por:

\begin{equation}
\lambda(x) = a \cdot x + b
\label{eq:lambda_pixel}
\end{equation}

onde $a$ é o fator de dispersão (nm/pixel) e $b$ é o \textit{offset} de comprimento de onda.

O método de calibração do OSA Visível utiliza o princípio de Mínimos Quadrados (\textit{Least Squares}) para determinar $a$ e $b$, mas com uma ênfase particular na calibração por pontos fixos. Os lasers de referência Verde ($\lambda_{\textrm{verd}} = 532$ nm) e Vermelho ($\lambda_{\textrm{verm}} = 650$ nm) fornecem dois pontos metrológicos de ancoragem absoluta.

Os coeficientes $a$ e $b$ são calculados diretamente a partir da posição dos picos de intensidade desses lasers ($x_{\textrm{verd}}$ e $x_{\textrm{verm}}$):

\begin{equation}
a = \frac{\lambda_{\textrm{verm}} - \lambda_{\textrm{verd}}}{x_{\textrm{verm}} - x_{\textrm{verd}}}, \quad b = \lambda_{\textrm{verd}} - a \cdot x_{\textrm{verd}}
\label{eq:coeficientes}
\end{equation}

Utilizar dois pontos fixos é fundamental. Um único ponto definiria apenas o \textit{offset} $b$, mas deixaria a taxa de dispersão $a$ (nm/pixel) sujeita a erros sistêmicos. A determinação de ambos os parâmetros por dois pontos de referência garante que o escalonamento da dispersão angular seja corretamente aplicado em toda a faixa de operação do espectrômetro, minimizando o erro de inclinação.

\subsection{\textbf{Metrologia de Acurácia: O Conceito de Erro RMS}}

A acurácia do mapeamento comprimento de onda-pixel é avaliada utilizando o Erro Quadrático Médio (\textit{Root Mean Square Error} - RMS), uma métrica padrão na metrologia espectral que quantifica a magnitude média do erro entre os valores medidos e os valores de referência \cite{henriksen_calibration}.

O processo híbrido de calibração do OSA Visível demonstrou uma melhoria significativa de precisão: a calibração preliminar baseada em luz branca resultou em um erro RMS de $\pm 2.1$ nm, que foi subsequentemente refinado para \textbf{$\pm 1.8$ nm} após o ajuste absoluto utilizando os lasers de referência.

O erro RMS final de $\pm 1.8$ nm valida a escolha do modelo de primeira ordem para esta aplicação. Embora não atinja a precisão de sub-picômetros de OSAs comerciais sofisticados, esta acurácia é mais do que suficiente para distinguir picos espectrais largos na faixa visível e satisfazer os requisitos de aplicações quantitativas baseadas na Lei de Beer-Lambert, como comprovado pelos altos coeficientes de determinação ($R^2$) nos ensaios com glicerina.

A Tabela \ref{tab:parametros_calibracao} apresenta os parâmetros chave da calibração linear do OSA Visível.

\begin{table}[ht]
\centering
\caption{Parâmetros Chave da Calibração Linear do OSA Visível}
\label{tab:parametros_calibracao}
\begin{tabular}{>{\raggedright\arraybackslash}p{3.5cm}>{\centering\arraybackslash}p{3cm}>{\centering\arraybackslash}p{3cm}>{\raggedright\arraybackslash}p{3.5cm}}
\hline
Parâmetro Metrológico & Ajuste Preliminar (Luz Branca) & Ajuste Final (Lasers 532/650 nm) & Significado Físico \\
\hline
Coeficiente $a$ (nm/pixel) & -0.060825 & 1.475 & Taxa de dispersão (nm por pixel) \\
Coeficiente $b$ (nm) & 203.1368 & 195.7 & Intercepto (Wavelength Offset) \\
Erro RMS & $\pm 2.1$ nm & $\pm 1.8$ nm & Acurácia final da calibração \\
\hline
\end{tabular}
\fonte{Elaborada pelo autor (2025).}
\end{table}

\section{APLICAÇÃO DE VISÃO COMPUTACIONAL PARA CALIBRAÇÃO AUTOMATIZADA}

A Visão Computacional é o diferencial tecnológico que permite ao OSA Visível executar calibrações de forma automatizada, robusta e precisa, substituindo métodos manuais demorados ou dependentes de hardware dedicado.

\subsection{\textbf{Processamento Digital do Sinal Espectral (Imagens CMOS)}}

O \textit{pipeline} de processamento digital é crucial para extrair o dado espectral da imagem bruta capturada pela webcam. Uma etapa inicial essencial é a Média Temporal (\textit{Frame Averaging}). Em sensores CMOS de baixo custo, o ruído aleatório (como ruído de disparo ou ruído térmico) é significativo. Para a calibração com lasers, o software captura $N=20$ quadros em um curto intervalo e calcula a intensidade média $\bar{I}(x,y)$:

\begin{equation}
\bar{I}(x,y) = \frac{1}{N}\sum_{i=1}^{N} I_i(x,y)
\label{eq:frame_avg}
\end{equation}

O cálculo da média temporal melhora drasticamente a Relação Sinal-Ruído (SNR), permitindo a detecção mais precisa dos picos de intensidade dos lasers. Após a média, a imagem é convertida para escala de cinza, simplificando a análise da intensidade luminosa.

\subsection{\textbf{Segmentação e Limiarização Adaptativa de Imagem (Método de Otsu)}}

Para isolar o espectro (o sinal de interesse) do fundo escuro e do ruído residual da imagem, é necessária uma etapa de segmentação. O \textbf{Método de Otsu} é um algoritmo automático de limiarização global que opera maximizando a variância entre classes (\textit{foreground} vs. \textit{background}) no histograma de intensidade da imagem \cite{otsu1979}.

O Otsu calcula o limiar ótimo $T$ que melhor divide os pixels da imagem em duas classes, garantindo que o sinal espectral, mesmo que difuso, seja completamente capturado. Isso resulta em uma máscara binária $M(x,y)$:

\begin{equation}
M(x,y) = \begin{cases}
I_{gray}(x,y), & \textrm{se } I_{gray}(x,y) \geq T \\
0, & \textrm{caso contrário}
\end{cases}
\label{eq:threshold}
\end{equation}

A aplicação de um limiar adaptativo torna o processo de calibração robusto a variações na iluminação da fonte de luz branca e às condições ambientais, eliminando a necessidade de ajustes manuais do limiar.

\subsection{\textbf{Localização de Fontes: Centróide e Detecção de Picos}}

A visão computacional é usada para duas tarefas críticas de localização: o centróide para o espectro contínuo e a detecção de pico para os lasers de referência.

\textbf{Cálculo do Centróide:} O centróide $(x_c, y_c)$ representa o centro de massa ponderado pela intensidade dos pixels ativos dentro da máscara $M$ \cite{spectral_centroid_wiki}:

\begin{equation}
x_c = \frac{\sum_{x,y} x \cdot M(x,y)}{\sum_{x,y} M(x,y)}, \quad y_c = \frac{\sum_{x,y} y \cdot M(x,y)}{\sum_{x,y} M(x,y)}
\label{eq:centroid}
\end{equation}

Este cálculo é vital, pois o espectrômetro "Osinha", devido à sua construção 3D com tolerâncias inerentes, pode apresentar um desalinhamento rotacional do espectro no sensor CMOS. O centróide fornece um ponto de referência central que permite à regressão linear preliminar definir corretamente a orientação espacial da linha espectral na imagem (a inclinação da reta $y=mx+c$), compensando o desalinhamento mecânico antes da calibração absoluta.

\textbf{Detecção de Picos:} Para os lasers de referência, a posição de comprimento de onda é determinada pelo pixel de máxima intensidade. A posição do pico ($x_{peak}$) é determinada pelo argumento máximo da função intensidade $I(x)$ ao longo da linha espectral:

\begin{equation}
x_{peak} = \arg \max_{x} \, I(x)
\label{eq:peak}
\end{equation}

Esta precisão na localização dos picos é o que garante a acurácia metrológica final da calibração, fornecendo os pontos fixos necessários para a determinação dos coeficientes $a$ e $b$.

A Tabela \ref{tab:algoritmos_visao} resume os algoritmos de visão computacional aplicados à calibração.

\begin{table}[ht]
\centering
\caption{Algoritmos de Visão Computacional Aplicados à Calibração}
\label{tab:algoritmos_visao}
\begin{tabular}{>{\raggedright\arraybackslash}p{3cm}>{\raggedright\arraybackslash}p{3cm}>{\raggedright\arraybackslash}p{3cm}>{\raggedright\arraybackslash}p{3.5cm}}
\hline
Algoritmo (Referência) & Propósito & Impacto Metrológico & Justificativa para Baixo Custo \\
\hline
Média Temporal & Aumento do SNR & Reduz erro aleatório na intensidade & Mitiga o ruído inerente a sensores CMOS de baixo custo \\
Limiarização Otsu \cite{otsu1979} & Segmentação ROI espectral & Isolamento do sinal de interesse & Automatiza o processo, tornando-o robusto a variações de iluminação \\
Cálculo de Centróide \cite{spectral_centroid_wiki} & Localização geométrica & Define a orientação espacial da dispersão & Compensa desalinhamentos mecânicos da estrutura 3D \\
Regressão Linear & Mapeamento $\lambda(x)$ & Converte pixel para nm com precisão funcional & Eficácia metrológica com baixo custo computacional \\
\hline
\end{tabular}
\fonte{Elaborada pelo autor (2025).}
\end{table}

\section{O PROCESSO HÍBRIDO DE CALIBRAÇÃO DO OSA VISÍVEL}

O processo de calibração do "OSA Visível" é um método híbrido que combina a inteligência da visão computacional com referências metrológicas de comprimento de onda conhecido, garantindo a acurácia dentro do limite de custo estabelecido.

\subsection{\textbf{Etapa 1: Calibração Preliminar com Luz Branca (Ajuste da Orientação Espacial)}}

O primeiro passo utiliza o espectro contínuo de uma fonte de luz branca. O objetivo desta fase é obter um ajuste inicial do espectro na imagem, crucial para compensar a geometria imperfeita da montagem 3D.

O processo envolve as seguintes etapas:

\begin{enumerate}
    \item O software captura um quadro, converte-o para escala de cinza e aplica o método de Otsu para segmentação.
    \item É calculado o centróide $(x_c, y_c)$ da nuvem de pixels ativos. A localização do centróide fornece o ponto central do espectro na imagem.
    \item Uma regressão linear é aplicada à nuvem de pontos ativa para determinar a inclinação da reta espectral ($y = m x + c$). Essa reta define o eixo de dispersão.
    \item Os parâmetros preliminares $a$ e $b$ da relação $\lambda(x)$ são calculados com base nesta orientação inicial, resultando em um erro RMS de $\pm 2.1$ nm. Esta etapa estabelece uma correlação linear entre a intensidade dos pixels e sua posição na imagem.
\end{enumerate}

\subsection{\textbf{Etapa 2: Calibração Absoluta com Lasers de Comprimento de Onda Conhecido}}

A Etapa 2 é a fase de ancoragem metrológica. Após a calibração inicial que alinha espacialmente o espectro, os lasers de referência são utilizados para mapear essa reta de dispersão preliminar aos comprimentos de onda absolutos conhecidos.

São utilizados dois lasers, Verde (532 nm) e Vermelho (650 nm). O processo compreende:

\begin{enumerate}
    \item O software captura 20 quadros e aplica a média temporal para maximizar o SNR e a precisão.
    \item A posição exata dos picos de intensidade dos lasers ($x_{532}$ e $x_{650}$) é determinada pelo cálculo do Arg Max.
    \item Utilizando as coordenadas absolutas (pico de pixel, comprimento de onda real), o software calcula os coeficientes finais $a$ e $b$ que definem a relação $\lambda(x) = a x + b$.
\end{enumerate}

Essa correspondência entre as posições dos picos na reta de calibração e os comprimentos de onda reais garante que a dispersão (nm/pixel) seja corretamente escalada em toda a faixa de operação do OSA, culminando na precisão final de $\pm 1.8$ nm. A utilização de dois pontos fixos é essencial para definir inequivocamente tanto o \textit{offset} quanto a dispersão da reta linear.

\subsection{\textbf{Validação Experimental do Sistema: Ensaio de Beer-Lambert}}

A validação experimental provou que a calibração automatizada é metrologicamente válida para aplicações quantitativas. O ensaio de diluição de glicerina em água (faixa de 10-30\% v/v) demonstrou que o sistema é capaz de medir a absorbância com alta fidelidade.

Os espectros de absorção obtidos mostraram picos característicos em 511 nm (verde) e 620 nm (vermelho). A correlação entre os dados experimentais de absorbância e as concentrações teóricas (aplicando o princípio Beer-Lambert) resultou em coeficientes de determinação ($R^2$) consistentemente altos, variando entre 0.93 e 0.99 para diferentes fontes de luz (LEDs azul, verde e vermelho) e espessuras de amostra. Este resultado valida empiricamente que a acurácia de $\pm 1.8$ nm obtida pela calibração híbrida é suficiente para o propósito de análise espectral de baixo custo.

A principal limitação desse modelo reside na escolha da regressão linear de primeira ordem, que, embora funcional na faixa visível, pode introduzir erros sistemáticos maiores nas extremidades do espectro (abaixo de 400 nm e acima de 700 nm), onde a não-linearidade da Equação de Difração de Fraunhofer se manifesta mais intensamente. Trabalhos futuros devem, portanto, explorar a incorporação de modelos não lineares (polinômios de ordem superior) para aprimorar a precisão nessas regiões. Outras avenidas de desenvolvimento incluem a expansão da faixa espectral para o infravermelho próximo (750-1100 nm), o que exigirá a adaptação do sensor CMOS e uma nova reavaliação dos parâmetros metrológicos e algorítmicos.

\chapter{METODOLOGIA DE DESENVOLVIMENTO DO HARDWARE E SOFTWARE}

Este capítulo detalha os procedimentos metodológicos adotados para o desenvolvimento do Analisador de Espectro Óptico (OSA) de baixo custo "Osinha" e do software "OSA Visível". A metodologia abrange desde a seleção e montagem dos componentes ópticos até os processos de calibração espectral automatizada via visão computacional, assegurando a precisão e confiabilidade das medições espectrais na faixa visível (380-750 nm).

\section{FUNDAMENTAÇÃO METODOLÓGICA E TRABALHOS CORRELATOS}

A calibração de analisadores de espectro óptico (OSA) para a faixa visível tem sido amplamente explorada na literatura científica. \cite{dubard1995} discutiram a necessidade de caracterizar com precisão as fontes ópticas em medições de fibras ópticas e apresentaram técnicas de calibração para OSAs comerciais. Abordagens modernas, como as propostas por \cite{liu2013}, empregam métodos de calibração precisos utilizando parâmetros do sistema para espectrômetros de grade. Alternativamente, \cite{terra_calibration} propuseram três métodos para calibrar um OSA baseado em grade, utilizando fontes laser bem caracterizadas, duplicação de frequência em cristais não lineares e uma célula de gás de cianeto de hidrogênio (HCN) como referência de comprimento de onda.

No contexto de espectrômetros de baixo custo, trabalhos recentes têm demonstrado a viabilidade de utilizar componentes comerciais prontos para uso (COTS) e técnicas de fabricação aditiva. \cite{gaudi_spectrometer} desenvolveram o projeto \textit{Open Fiber Spectrometer}, um espectrômetro de código aberto que fornece modelos 3D e instruções detalhadas para construção, servindo como base para diversos projetos educacionais e de pesquisa.

A metodologia adotada neste trabalho diferencia-se das abordagens existentes por combinar uma estratégia híbrida de calibração: a calibração inicial com luz branca fornece uma relação linear preliminar, enquanto lasers de referência (532 nm e 650 nm) definem a escala absoluta, combinando assim simplicidade e baixo custo — uma lacuna não abordada pelos métodos existentes. Esta abordagem permite alcançar precisão metrológica adequada ($\pm 1.8$ nm) a um custo total inferior a \$200, democratizando o acesso à análise espectral.

\section{CONSTRUÇÃO DO HARDWARE: ESPECTRÔMETRO "OSINHA"}

\subsection{Seleção e Especificação dos Componentes}

O hardware do espectrômetro "Osinha" foi desenvolvido com base no projeto \textit{Open Fiber Spectrometer} \cite{gaudi_spectrometer}, um espectrômetro de código aberto que fornece modelos 3D e instruções detalhadas para construção. O projeto original foi adaptado para atender às necessidades específicas deste trabalho, com foco na faixa visível do espectro (380-750 nm).

A seleção dos componentes foi orientada por critérios de baixo custo, disponibilidade comercial e adequação às especificações técnicas necessárias para operação na faixa visível. A Tabela \ref{tab:componentes} apresenta os componentes principais e suas especificações.

\begin{table}[ht]
\centering
\caption{Componentes principais do espectrômetro "Osinha" e especificações técnicas}
\label{tab:componentes}
\begin{tabular}{>{\raggedright\arraybackslash}p{4cm}>{\raggedright\arraybackslash}p{3cm}>{\raggedright\arraybackslash}p{3cm}>{\raggedright\arraybackslash}p{2.5cm}}
\hline
Componente & Especificação & Função & Custo Aprox. (USD) \\
\hline
Estrutura 3D & PLA, impressão FDM & Suporte mecânico e alinhamento óptico & 5-10 \\
Grade de Difração & 1000 linhas/mm, transmissão & Dispersão espectral & 50-80 \\
Webcam USB & 640$\times$480 pixels, CMOS & Detecção espectral & 20-40 \\
Fonte de Luz Branca & LED RGB, CRI 82 & Calibração preliminar & 10-15 \\
Lasers de Referência & 532 nm (verde), 650 nm (vermelho) & Calibração absoluta & 30-50 \\
Fenda Óptica & Ajustável, 50-200 $\mu$m & Controle de resolução & 5-10 \\
\hline
\textbf{Total} & & & \textbf{< \$200} \\
\hline
\end{tabular}
\fonte{Elaborada pelo autor (2025).}
\end{table}

\subsection{Arquitetura Óptica e Princípios de Funcionamento}

A arquitetura óptica do espectrômetro "Osinha" segue o princípio clássico de um espectrômetro de grade de transmissão, adaptado para componentes de baixo custo. A Figura \ref{fig:osinha} ilustra a disposição dos componentes principais.

\begin{figure}[ht]
\centering
\includegraphics[width=0.6\textwidth]{../SBAI_SBSE_2025___OSA_Visível___Review/data/Osinha - PAPER.png}
\caption{Arquitetura do espectrômetro "Osinha": (1) Luz incidente, (2) Fenda óptica, (3) Grade de difração, (4) Webcam CMOS. Adaptado de \cite{gaudi_spectrometer}.}
\label{fig:osinha}
\end{figure}

O caminho óptico segue a seguinte sequência:

\begin{enumerate}
    \item \textbf{Entrada de Luz}: A luz proveniente da amostra ou fonte de calibração entra através de uma fenda óptica ajustável, que controla a largura do feixe incidente e, consequentemente, a resolução espectral do sistema.
    
    \item \textbf{Dispersão Espectral}: A luz colimada incide sobre uma grade de difração de transmissão com densidade de 1000 linhas/mm. A relação entre o comprimento de onda ($\lambda$) e o ângulo de difração ($\theta$) é governada pela equação da grade de Fraunhofer:
    
    \begin{equation}
        n\lambda = d(\sin\theta + \sin\alpha)
        \label{eq:diffraction_metodologia}
    \end{equation}
    
    onde $n$ representa a ordem de difração (geralmente $n=1$), $d = 1\,\mu\text{m}$ é o espaçamento da grade (inverso da densidade de linhas), e $\alpha$ é o ângulo de incidência da luz na grade \cite{fraunhofer_wiki}.
    
    \item \textbf{Detecção}: O espectro disperso é capturado por um sensor CMOS de uma webcam USB, posicionado no plano focal da grade. O sensor discretiza a informação espectral em uma matriz de pixels, onde cada coluna corresponde a uma faixa de comprimentos de onda.
\end{enumerate}

\subsection{Fabricação da Estrutura por Impressão 3D}

A estrutura física do espectrômetro foi projetada utilizando software de modelagem 3D (CAD) e fabricada por meio de impressão 3D utilizando tecnologia de modelagem por deposição fundida (FDM - \textit{Fused Deposition Modeling}). Esta abordagem oferece várias vantagens estratégicas:

\begin{itemize}
    \item \textbf{Custo Reduzido}: A impressão 3D elimina a necessidade de ferramentas especializadas e processos de usinagem de precisão, reduzindo drasticamente os custos de fabricação \cite{stanford_3d_optics}.
    
    \item \textbf{Personalização}: O design pode ser facilmente modificado e adaptado para diferentes aplicações ou requisitos específicos, sem custos adicionais significativos.
    
    \item \textbf{Reprodutibilidade}: Os arquivos de modelo 3D podem ser compartilhados e replicados em qualquer local com acesso a uma impressora 3D, facilitando a disseminação do projeto.
    
    \item \textbf{Integração de Componentes}: A estrutura impressa incorpora suportes e guias para posicionamento preciso dos componentes ópticos, garantindo alinhamento adequado sem necessidade de ajustes mecânicos complexos.
\end{itemize}

O material utilizado foi filamento de ácido polilático (PLA), escolhido por sua estabilidade dimensional, facilidade de impressão e baixo custo. A resolução de impressão foi configurada para 0,2 mm de altura de camada, garantindo superfícies suficientemente lisas para aplicações ópticas de baixa precisão.

\subsection{Montagem e Alinhamento Óptico}

O processo de montagem seguiu uma sequência sistemática para garantir o alinhamento adequado dos componentes:

\begin{enumerate}
    \item \textbf{Instalação da Estrutura Base}: A estrutura impressa em 3D foi montada e verificada quanto à integridade estrutural e dimensional.
    
    \item \textbf{Posicionamento da Fenda Óptica}: A fenda foi instalada na entrada óptica e ajustada para uma largura de aproximadamente 100 $\mu$m, otimizando o compromisso entre resolução espectral e intensidade do sinal.
    
    \item \textbf{Instalação da Grade de Difração}: A grade foi posicionada perpendicularmente ao eixo óptico, com o ângulo de incidência ($\alpha$) ajustado para aproximadamente 45$^\circ$, garantindo dispersão adequada na faixa visível.
    
    \item \textbf{Posicionamento do Detector}: A webcam foi fixada na estrutura de forma que o sensor CMOS capture todo o espectro disperso, com o eixo horizontal do sensor alinhado com a direção de dispersão.
    
    \item \textbf{Verificação do Alinhamento}: O alinhamento foi verificado visualmente utilizando uma fonte de luz branca, observando a formação de um espectro contínuo e uniforme no sensor.
\end{enumerate}

\subsection{Caracterização do Sistema de Detecção}

O sistema de detecção baseado em webcam USB apresenta características específicas que influenciam o desempenho do espectrômetro. O sensor CMOS utilizado possui resolução de $640 \times 480$ pixels, com taxa de amostragem de 30 fps. A discretização do sinal óptico é representada pela equação:

\begin{equation}
    I(x,y) = \sum_{k=0}^{255} k \cdot P(k|x,y)
    \label{eq:sampling_metodologia}
\end{equation}

onde $I(x,y)$ é a intensidade digital no pixel $(x,y)$, $k$ representa os níveis de intensidade (0 a 255 em escala de 8 bits), e $P(k|x,y)$ é a probabilidade do pixel $(x,y)$ registrar a intensidade $k$.

Uma característica importante do sensor CMOS é a presença de uma matriz de filtros Bayer (RGB), que significa que cada pixel é sensível a uma faixa específica do espectro (vermelho, verde ou azul). Para a calibração espectral, esta característica é contornada através da conversão para escala de cinza, que integra a resposta espectral dos três canais de cor.

\section{DESENVOLVIMENTO DO SOFTWARE "OSA VISÍVEL"}

\subsection{Arquitetura e Plataforma de Desenvolvimento}

O software "OSA Visível" foi desenvolvido em Python 3.8+, escolhido por sua portabilidade, vasta gama de bibliotecas científicas e facilidade de integração com hardware. A arquitetura do software segue um modelo modular, separando as funcionalidades de aquisição de dados, processamento de imagem, calibração e visualização.

As principais bibliotecas utilizadas incluem:

\begin{itemize}
    \item \textbf{OpenCV (cv2)}: Para aquisição de vídeo, processamento de imagem e implementação de algoritmos de visão computacional.
    
    \item \textbf{Pillow (PIL)}: Para manipulação adicional de imagens e conversão de formatos.
    
    \item \textbf{NumPy}: Para operações numéricas e manipulação de arrays multidimensionais.
    
    \item \textbf{SciPy}: Para funções científicas avançadas, incluindo regressão linear e processamento de sinais.
    
    \item \textbf{Matplotlib}: Para visualização de gráficos e espectros.
    
    \item \textbf{Tkinter}: Para desenvolvimento da interface gráfica do usuário (GUI).
\end{itemize}

A compatibilidade multiplataforma (Windows 10/11 e Linux Ubuntu 22.04) foi garantida através do uso de bibliotecas padrão e testes extensivos em ambos os ambientes.

\subsection{Interface Gráfica do Usuário}

A interface gráfica foi projetada para guiar o usuário através de um fluxo intuitivo de operação, dispensando conhecimentos avançados em programação. A Figura \ref{fig:gui_principal} ilustra a janela principal do software.

\begin{figure}[ht]
\centering
\includegraphics[width=0.7\textwidth]{../SBAI_SBSE_2025___OSA_Visível___Review/data/janela_principal_recursos.png}
\caption{Janela principal do software "OSA Visível": (1) Visualização do espectro em tempo real, (2) Gráfico de intensidade $I(\lambda)$ vs. comprimento de onda, (3) Painel de controle e calibração.}
\label{fig:gui_principal}
\end{figure}

As principais funcionalidades da interface incluem:

\begin{itemize}
    \item \textbf{Visualização em Tempo Real}: Exibição contínua do espectro capturado pela webcam, permitindo monitoramento visual da qualidade do sinal.
    
    \item \textbf{Gráficos Espectrais}: Visualização dos dados processados em formato de gráfico $I(\lambda)$, com opções de zoom e análise de picos.
    
    \item \textbf{Painel de Calibração}: Acesso rápido às ferramentas de calibração, com assistente passo a passo para guiar o usuário.
    
    \item \textbf{Exportação de Dados}: Funcionalidade para salvar espectros em formato texto (.txt) para análise posterior.
    
    \item \textbf{Configurações do Sistema}: Ajustes de parâmetros da webcam, seleção de dispositivo de vídeo e configurações de processamento.
\end{itemize}

\section{PROCESSO DE CALIBRAÇÃO AUTOMATIZADA}

A calibração do espectrômetro é o processo crítico que estabelece a correspondência precisa entre a posição dos pixels no sensor CMOS e os comprimentos de onda correspondentes. O método desenvolvido combina técnicas de visão computacional com referências metrológicas de comprimento de onda conhecido, resultando em um processo híbrido automatizado.

\subsection{Etapa 1: Calibração Preliminar com Luz Branca}

A primeira etapa do processo de calibração utiliza o espectro contínuo de uma fonte de luz branca para estabelecer uma relação preliminar entre posição de pixel e comprimento de onda. Esta etapa é fundamental para compensar desalinhamentos mecânicos e definir a orientação espacial do espectro na imagem.

\subsubsection{Aquisição do Sinal Digital}

A calibração inicia com a captura de um único quadro ($N=1$) do espectro contínuo via biblioteca OpenCV. A escolha de um único quadro é adequada para esta etapa, pois o espectro contínuo de luz branca apresenta alta intensidade e baixa variabilidade temporal, não requerendo média temporal para redução de ruído.

O processo de aquisição segue os seguintes passos:

\begin{enumerate}
    \item Inicialização da webcam através da interface OpenCV.
    
    \item Configuração dos parâmetros de captura: resolução (640$\times$480), formato de cor (RGB), e taxa de quadros (30 fps).
    
    \item Captura de um quadro estático do espectro, armazenado como array NumPy de dimensões $(480, 640, 3)$, representando altura, largura e canais de cor (RGB).
\end{enumerate}

\subsubsection{Pré-processamento com Visão Computacional}

O pré-processamento da imagem é essencial para isolar o sinal espectral do fundo e do ruído, preparando os dados para análise quantitativa.

\textbf{Conversão para Escala de Cinza:}

A imagem RGB capturada é convertida para escala de cinza utilizando a média ponderada dos canais de cor:

\begin{equation}
    I_{gray}(x,y) = \frac{R(x,y) + G(x,y) + B(x,y)}{3}
    \label{eq:grayscale}
\end{equation}

onde $R(x,y)$, $G(x,y)$ e $B(x,y)$ são as intensidades dos canais vermelho, verde e azul no pixel $(x,y)$, respectivamente. Esta conversão simplifica a análise posterior, integrando a resposta espectral dos três canais de cor do sensor Bayer.

\textbf{Segmentação por Limiarização Adaptativa:}

Para isolar o espectro (sinal de interesse) do fundo escuro e do ruído residual, é aplicada uma etapa de segmentação utilizando o método de Otsu \cite{otsu1979}. Este método calcula automaticamente o limiar ótimo $T$ que maximiza a variância entre classes (foreground vs. background) no histograma de intensidade da imagem.

O método de Otsu opera da seguinte forma:

\begin{enumerate}
    \item Cálculo do histograma de intensidades da imagem em escala de cinza.
    
    \item Para cada possível valor de limiar $t$ (0 a 255), cálculo da variância entre classes:
    
    \begin{equation}
        \sigma_B^2(t) = \omega_0(t)\omega_1(t)[\mu_0(t) - \mu_1(t)]^2
    \end{equation}
    
    onde $\omega_0(t)$ e $\omega_1(t)$ são as probabilidades das classes (fundo e objeto), e $\mu_0(t)$ e $\mu_1(t)$ são as médias das classes.
    
    \item Seleção do limiar $T$ que maximiza $\sigma_B^2(t)$.
\end{enumerate}

A aplicação do limiar resulta em uma máscara binária $M(x,y)$:

\begin{equation}
    M(x,y) = \begin{cases}
        I_{gray}(x,y), & \text{se } I_{gray}(x,y) \geq T \\
        0, & \text{caso contrário}
    \end{cases}
    \label{eq:threshold_metodologia}
\end{equation}

A utilização de um limiar adaptativo torna o processo robusto a variações na iluminação da fonte de luz branca e às condições ambientais, eliminando a necessidade de ajustes manuais.

\subsubsection{Cálculo do Centróide e Regressão Linear Preliminar}

\textbf{Cálculo do Centróide:}

O centróide $(x_c, y_c)$ representa o centro de massa ponderado pela intensidade dos pixels ativos dentro da máscara $M$ \cite{spectral_centroid_wiki}:

\begin{equation}
    x_c = \frac{\sum_{x,y} x \cdot M(x,y)}{\sum_{x,y} M(x,y)}, \quad
    y_c = \frac{\sum_{x,y} y \cdot M(x,y)}{\sum_{x,y} M(x,y)}
    \label{eq:centroid_metodologia}
\end{equation}

Este cálculo é vital, pois o espectrômetro "Osinha", devido à sua construção 3D com tolerâncias inerentes, pode apresentar um desalinhamento rotacional do espectro no sensor CMOS. O centróide fornece um ponto de referência central que permite à regressão linear preliminar definir corretamente a orientação espacial da linha espectral na imagem.

\textbf{Regressão Linear para Orientação Espacial:}

Após a segmentação, os pixels ativos ($M(x,y) \geq T$) formam uma nuvem de pontos ao longo do espectro. Uma regressão linear é aplicada para determinar a inclinação da reta espectral ($y = m \cdot x + c$), que define o eixo de dispersão.

A regressão linear é obtida via método dos mínimos quadrados, minimizando:

\begin{equation}
    \min_{m,c} \sum_{x,y} \left(y - (m \cdot x + c)\right)^2
    \label{eq:regression_metodologia}
\end{equation}

Os coeficientes $m$ (inclinação) e $c$ (intercepto) definem a orientação espacial do espectro. Combinado com o centróide, esta reta estabelece uma relação preliminar entre posição de pixel e comprimento de onda, que será posteriormente ajustada pelos lasers de referência.

Os parâmetros preliminares $a$ e $b$ da relação $\lambda(x) = a \cdot x + b$ são calculados com base nesta orientação inicial, resultando em um erro RMS de $\pm 2.1$ nm. Esta etapa estabelece uma correlação linear entre a intensidade dos pixels e sua posição na imagem, servindo como base para a calibração absoluta subsequente.

\subsection{Etapa 2: Calibração Absoluta com Lasers de Referência}

A segunda etapa do processo de calibração é a fase de ancoragem metrológica. Após a calibração inicial que alinha espacialmente o espectro, os lasers de referência são utilizados para mapear essa reta de dispersão preliminar aos comprimentos de onda absolutos conhecidos.

\subsubsection{Seleção das Fontes de Referência}

São utilizados dois lasers de comprimento de onda conhecido:

\begin{itemize}
    \item \textbf{Laser Verde}: Comprimento de onda $\lambda_{\text{verd}} = 532$ nm, correspondente ao segundo harmônico de um laser Nd:YAG.
    
    \item \textbf{Laser Vermelho}: Comprimento de onda $\lambda_{\text{verm}} = 650$ nm, típico de diodos laser de baixa potência.
\end{itemize}

A escolha destes dois comprimentos de onda específicos é estratégica:

\begin{enumerate}
    \item \textbf{Cobertura Espectral}: Os dois pontos estão distribuídos ao longo da faixa visível, permitindo uma calibração linear adequada para toda a faixa de operação (380-750 nm).
    
    \item \textbf{Disponibilidade e Custo}: Lasers de 532 nm e 650 nm são amplamente disponíveis comercialmente a baixo custo, adequados para aplicações educacionais.
    
    \item \textbf{Precisão Metrológica}: Lasers de diodo apresentam estabilidade de comprimento de onda adequada ($\pm 1$ nm) para a precisão desejada do sistema.
\end{enumerate}

\subsubsection{Aquisição com Média Temporal}

Para a calibração utilizando lasers, $N=20$ quadros são capturados em um intervalo homogêneo de tempo (aproximadamente 2 segundos). A média temporal é aplicada para reduzir o ruído aleatório inerente a sensores CMOS de baixo custo:

\begin{equation}
    \bar{I}(x,y) = \frac{1}{N}\sum_{i=1}^{N} I_i(x,y)
    \label{eq:frame_avg_metodologia}
\end{equation}

O cálculo da média temporal melhora drasticamente a Relação Sinal-Ruído (SNR), permitindo a detecção mais precisa dos picos de intensidade dos lasers. Em sensores CMOS de baixo custo, o ruído aleatório (ruído de disparo, ruído térmico) é significativo, e a média temporal é uma técnica eficaz e computacionalmente simples para mitigá-lo.

\subsubsection{Detecção de Picos de Intensidade}

A posição exata dos picos de intensidade dos lasers ($x_{532}$ e $x_{650}$) é determinada pelo cálculo do argumento máximo (Arg Max) ao longo da linha espectral previamente identificada:

\begin{equation}
    x_{peak} = \arg \max_{x} \, I(x)
    \label{eq:peak_metodologia}
\end{equation}

onde $I(x)$ é a intensidade média ao longo da direção vertical (eixo $y$) para cada posição horizontal $x$:

\begin{equation}
    I(x) = \frac{1}{H}\sum_{y=0}^{H-1} \bar{I}(x,y)
\end{equation}

e $H$ é a altura da imagem (480 pixels).

Esta precisão na localização dos picos é o que garante a acurácia metrológica final da calibração, fornecendo os pontos fixos necessários para a determinação dos coeficientes $a$ e $b$.

\subsubsection{Cálculo dos Coeficientes de Calibração Final}

Utilizando as coordenadas absolutas (pico de pixel, comprimento de onda real), o software calcula os coeficientes finais $a$ e $b$ que definem a relação $\lambda(x) = a \cdot x + b$.

Os coeficientes são calculados diretamente a partir da posição dos picos de intensidade dos lasers:

\begin{equation}
    a = \frac{\lambda_{\text{verm}} - \lambda_{\text{verd}}}{x_{\text{verm}} - x_{\text{verd}}}, \quad
    b = \lambda_{\text{verd}} - a \cdot x_{\text{verd}}
    \label{eq:coeficientes_metodologia}
\end{equation}

onde $x_{\text{verd}}$ e $x_{\text{verm}}$ são as posições dos picos dos lasers verde e vermelho, respectivamente.

A utilização de dois pontos fixos é essencial para definir inequivocamente tanto o \textit{offset} ($b$) quanto a dispersão ($a$) da reta linear. Um único ponto definiria apenas o \textit{offset}, mas deixaria a taxa de dispersão sujeita a erros sistêmicos. A determinação de ambos os parâmetros por dois pontos de referência garante que o escalonamento da dispersão angular seja corretamente aplicado em toda a faixa de operação do espectrômetro, minimizando o erro de inclinação.

A Figura \ref{fig:calib_curve} ilustra a curva de calibração obtida experimentalmente.

\begin{figure}[ht]
\centering
\includegraphics[width=0.5\textwidth]{../SBAI_SBSE_2025___OSA_Visível___Review/data/calibration_curve.png}
\caption{Reta de calibração $\lambda(x) = a \cdot x + b$ ajustada via regressão linear utilizando lasers de referência (532 nm e 650 nm).}
\label{fig:calib_curve}
\end{figure}

\subsection{Validação da Calibração e Métricas de Precisão}

A acurácia do mapeamento comprimento de onda-pixel é avaliada utilizando o Erro Quadrático Médio (Root Mean Square Error - RMS), uma métrica padrão na metrologia espectral que quantifica a magnitude média do erro entre os valores medidos e os valores de referência \cite{henriksen_calibration}.

O processo híbrido de calibração do OSA Visível demonstrou uma melhoria significativa de precisão: a calibração preliminar baseada em luz branca resultou em um erro RMS de $\pm 2.1$ nm, que foi subsequentemente refinado para \textbf{$\pm 1.8$ nm} após o ajuste absoluto utilizando os lasers de referência.

A Tabela \ref{tab:calib_params_metodologia} apresenta os parâmetros de calibração obtidos experimentalmente.

\begin{table}[ht]
\centering
\caption{Parâmetros de calibração obtidos experimentalmente}
\label{tab:calib_params_metodologia}
\begin{tabular}{lcc}
\toprule
Parâmetro & Luz Branca & Lasers (Final) \\
\midrule
Coeficiente $a$ (nm/pixel) & -0.0608250474 & 1.475 \\
Coeficiente $b$ (nm) & 203.136815 & 195.7 \\
Erro RMS (nm) & $\pm 2.1$ & $\pm 1.8$ \\
\bottomrule
\end{tabular}
\fonte{Elaborada pelo autor (2025).}
\end{table}

O erro RMS final de $\pm 1.8$ nm valida a escolha do modelo de primeira ordem (regressão linear) para esta aplicação. Embora não atinja a precisão de sub-picômetros de OSAs comerciais sofisticados, esta acurácia é mais do que suficiente para distinguir picos espectrais largos na faixa visível e satisfazer os requisitos de aplicações quantitativas baseadas na Lei de Beer-Lambert.

\subsection{Interface de Calibração do Software}

A interface de calibração do software "OSA Visível" foi projetada para guiar o usuário através do processo de calibração de forma intuitiva e automatizada. A Figura \ref{fig:calib_window} ilustra a janela de calibração.

\begin{figure}[ht]
\centering
\includegraphics[width=0.5\textwidth]{../SBAI_SBSE_2025___OSA_Visível___Review/data/calibration_window.png}
\caption{Janela de calibração do software "OSA Visível": (1) Botão "Calibrar Luz Branca" para calibração automática preliminar, (2) Botões "Iniciar Laser (Verde/Vermelho)" para calibração absoluta passo a passo.}
\label{fig:calib_window}
\end{figure}

O fluxo de calibração automatizada inclui:

\begin{enumerate}
    \item \textbf{Calibração com Luz Branca}:
    \begin{itemize}
        \item O usuário posiciona uma fonte de luz branca na entrada do espectrômetro.
        \item Clica no botão "Calibrar Luz Branca".
        \item O software executa automaticamente: captura de quadro, pré-processamento, cálculo de centróide e regressão linear preliminar.
        \item Os parâmetros preliminares são salvos e exibidos na interface.
    \end{itemize}
    
    \item \textbf{Calibração com Lasers}:
    \begin{itemize}
        \item O usuário é guiado passo a passo para posicionar cada laser (verde e vermelho) sequencialmente.
        \item Para cada laser, o software captura 20 quadros, aplica média temporal e detecta o pico de intensidade.
        \item Após ambos os lasers serem calibrados, o software calcula automaticamente os coeficientes finais $a$ e $b$.
        \item A curva de calibração é exibida graficamente e os parâmetros finais são salvos.
    \end{itemize}
\end{enumerate}

\section{CONSIDERAÇÕES METODOLÓGICAS E LIMITAÇÕES}

A metodologia desenvolvida apresenta algumas limitações inerentes à escolha de componentes de baixo custo e à simplificação do modelo de calibração:

\begin{itemize}
    \item \textbf{Modelo Linear}: A escolha da regressão linear de primeira ordem, embora funcional na faixa visível, pode introduzir erros sistemáticos maiores nas extremidades do espectro (abaixo de 400 nm e acima de 700 nm), onde a não-linearidade da Equação de Difração de Fraunhofer se manifesta mais intensamente.
    
    \item \textbf{Resolução Espectral}: A resolução espectral do sistema é limitada pela resolução espacial do sensor CMOS (640 pixels) e pela largura da fenda óptica, resultando em uma resolução de aproximadamente 0.5 nm/pixel na faixa visível.
    
    \item \textbf{Estabilidade Temporal}: Componentes de baixo custo podem apresentar variações temporais de calibração devido a efeitos térmicos ou envelhecimento, requerendo recalibração periódica.
    
    \item \textbf{Resposta Espectral do Sensor}: A matriz de filtros Bayer do sensor CMOS introduz uma resposta espectral não-uniforme que, embora parcialmente compensada pela conversão para escala de cinza, pode afetar a precisão de medições de intensidade absoluta.
\end{itemize}

Trabalhos futuros devem explorar a incorporação de modelos não lineares (polinômios de ordem superior) para aprimorar a precisão nas extremidades do espectro, bem como técnicas de correção espectral mais sofisticadas para compensar a resposta não-uniforme do sensor.

\chapter{RESULTADOS EXPERIMENTAIS E VALIDAÇÃO}

Este capítulo apresenta os resultados experimentais obtidos com o OSA Visível desenvolvido neste trabalho, incluindo análise de repetibilidade temporal, comparação com um OSA comercial (ThorLabs) e validação estatística do sistema. Os experimentos foram conduzidos para avaliar a precisão, estabilidade e confiabilidade do sistema de baixo custo em condições de medição repetidas.

\section{DESCRIÇÃO DO SETUP EXPERIMENTAL}

Os experimentos foram realizados utilizando o espectrômetro "Osinha" (OSA Visível) desenvolvido conforme descrito no Capítulo 3, e um OSA comercial da ThorLabs (modelo OSA20xC) como referência para validação cruzada. % TODO: Adicionar figura do setup experimental (Figura \ref{fig:setup_experimental}) quando a imagem estiver disponível.

%\begin{figure}[ht]
%\centering
%\includegraphics[width=0.8\textwidth]{setup_experimental.png}
%\caption{Configuração experimental: (1) Espectrômetro "Osinha" (OSA Visível), (2) Fonte de luz com picos RGB característicos, (3) Sistema de acoplamento óptico, (4) Computador para aquisição e processamento de dados.}
%\label{fig:setup_experimental}
%\fonte{Elaborado pelo autor (2025).}
%\end{figure}

O setup experimental foi configurado em um ambiente controlado, minimizando variações de temperatura e iluminação ambiente. A fonte de luz utilizada consistiu em uma fonte de LED RGB com picos espectrais característicos nas faixas do azul (aproximadamente 460 nm), verde (aproximadamente 520 nm) e vermelho (aproximadamente 640 nm). O acoplamento óptico foi mantido constante durante todas as medições para garantir reprodutibilidade.

Para os experimentos temporais, a fonte de luz foi mantida em operação contínua e estável, enquanto o espectrômetro realizava aquisições sequenciais de espectros. O intervalo de tempo entre aquisições foi configurado para 10 segundos, permitindo capturar possíveis variações temporais no sistema enquanto mantinha a fonte de luz em regime estacionário.

\section{EXPERIMENTO 1: ANÁLISE DE REPETIBILIDADE TEMPORAL COM OSA VISÍVEL}

\subsection{\textbf{Procedimento Experimental}}

O primeiro experimento visou avaliar a repetibilidade e estabilidade do OSA Visível através da aquisição de uma série temporal de espectros. O procedimento experimental seguiu os seguintes passos:

\begin{enumerate}
    \item \textbf{Calibração Inicial:} O OSA Visível foi calibrado utilizando o processo híbrido descrito no Capítulo 3, envolvendo calibração preliminar com luz branca seguida de calibração absoluta com lasers de referência (532 nm e 650 nm).
    
    \item \textbf{Aquisição Temporal:} Após a calibração, uma fonte de luz com espectro característico (LED RGB) foi posicionada na entrada do espectrômetro. O sistema foi configurado para realizar aquisições automáticas a cada 10 segundos, totalizando 100 amostras espectrais.
    
    \item \textbf{Processamento e Análise:} Cada espectro adquirido foi processado automaticamente pelo software "OSA Visível", extraindo os dados de intensidade em função do comprimento de onda. Os espectros foram então analisados utilizando algoritmos de detecção de picos implementados em Python, baseados em \texttt{scipy.signal.find\_peaks} com parâmetros de \textit{prominence} e \textit{distance} otimizados para a faixa visível.
    
    \item \textbf{Agrupamento de Picos:} Os picos detectados em cada espectro foram agrupados utilizando clustering hierárquico (\texttt{scipy.cluster.hierarchy}), permitindo identificar picos correspondentes entre as diferentes amostras e calcular estatísticas de repetibilidade.
\end{enumerate}

\subsection{\textbf{Resultados e Análise Estatística}}

A análise estatística dos 100 espectros temporais revelou a presença de múltiplos picos espectrais, dos quais três foram identificados como os picos principais RGB. A Tabela \ref{tab:resultados_visible_osa} apresenta as estatísticas dos 3 picos principais identificados.

\begin{table}[ht]
\centering
\caption{Estatísticas dos 3 picos principais RGB identificados pelo OSA Visível (100 amostras temporais)}
\label{tab:resultados_visible_osa}
\begin{tabular}{lcccccc}
\toprule
Pico RGB & $\lambda$ médio (nm) & $\sigma$ (nm) & Incerteza expandida (nm) & Taxa de detecção (\%) & Intensidade média & CV intensidade (\%) \\
\midrule
RGB-2 (Azul) & 468,22 & 1,469 & $\pm$0,288 & 100,0 & 122,12 & 1,12 \\
RGB-1 (Verde) & 516,10 & 0,682 & $\pm$0,134 & 100,0 & 177,85 & 0,61 \\
RGB-3 (Vermelho) & 637,70 & 0,701 & $\pm$0,143 & 92,0 & 84,23 & 0,38 \\
\bottomrule
\end{tabular}
\fonte{Elaborada pelo autor (2025).}
\end{table}

Os resultados demonstram excelente estabilidade e repetibilidade do sistema. O pico verde apresentou a maior estabilidade, com desvio padrão de apenas 0,682 nm e taxa de detecção de 100\% em todas as 100 amostras. O pico azul, embora tenha apresentado desvio padrão ligeiramente maior (1,469 nm), também foi detectado em 100\% das amostras, confirmando a robustez do sistema de detecção.

O pico vermelho apresentou taxa de detecção de 92\% (92 das 100 amostras), indicando que este pico pode estar próximo do limite de sensibilidade do sistema ou sujeito a variações mais significativas. No entanto, quando detectado, apresentou desvio padrão de apenas 0,701 nm, comparável ao pico verde.

As incertezas expandidas (calculadas com fator de cobertura $k=1,96$ para nível de confiança de 95\%) variaram entre $\pm$0,134 nm (verde) e $\pm$0,288 nm (azul), valores que estão dentro do objetivo de precisão estabelecido no início deste trabalho ($\pm$2 nm).

O coeficiente de variação (CV) das intensidades foi inferior a 2\% para todos os picos, indicando excelente estabilidade na medição de potência óptica. Este resultado valida não apenas a calibração do eixo de comprimento de onda ($\lambda$), mas também a precisão na medição da intensidade (eixo Y do espectro).

\section{EXPERIMENTO 2: VALIDAÇÃO CRUZADA COM OSA COMERCIAL (THORLABS)}

\subsection{\textbf{Procedimento Experimental}}

Para validar a acurácia do OSA Visível e compará-lo com padrões metrológicos estabelecidos, foi realizado um experimento comparativo utilizando um OSA comercial da ThorLabs (modelo OSA20xC). O procedimento seguiu os seguintes passos:

\begin{enumerate}
    \item \textbf{Seleção de Dados:} Foram utilizados dados temporais coletados previamente com o OSA ThorLabs, totalizando mais de 60.000 espectros adquiridos ao longo de aproximadamente 1000 segundos.
    
    \item \textbf{Amostragem Representativa:} Dado que o objetivo era avaliar repetibilidade com 100 amostras espaçadas de 10 segundos (equivalente ao experimento com OSA Visível), foram selecionados 100 espectros do conjunto temporal do ThorLabs, espaçados uniformemente no tempo para garantir representatividade estatística.
    
    \item \textbf{Conversão e Normalização:} Os espectros do ThorLabs foram convertidos para o formato padronizado utilizado pelo OSA Visível (arquivos de texto com colunas: comprimento de onda em metros; intensidade), garantindo que ambas as análises fossem realizadas com os mesmos algoritmos e parâmetros.
    
    \item \textbf{Análise Comparativa:} Os espectros foram processados utilizando o mesmo \textit{pipeline} de análise desenvolvido para o OSA Visível, permitindo comparação direta dos resultados.
\end{enumerate}

\subsection{\textbf{Resultados e Comparação Estatística}}

A análise dos 100 espectros selecionados do ThorLabs OSA também identificou os três picos principais RGB. A Tabela \ref{tab:resultados_thorlabs} apresenta as estatísticas obtidas.

\begin{table}[ht]
\centering
\caption{Estatísticas dos 3 picos principais RGB identificados pelo ThorLabs OSA (100 amostras selecionadas)}
\label{tab:resultados_thorlabs}
\begin{tabular}{lcccccc}
\toprule
Pico RGB & $\lambda$ médio (nm) & $\sigma$ (nm) & Incerteza expandida (nm) & Taxa de detecção (\%) & Intensidade média & CV intensidade (\%) \\
\midrule
RGB-1 (Azul) & 459,01 & 1,400 & $\pm$0,137 & 400,0 & 13915,96 & 3,44 \\
RGB-2 (Verde) & 519,05 & 1,467 & $\pm$0,150 & 369,0 & 6322,13 & 1,31 \\
RGB-3 (Vermelho) & 639,13 & 1,419 & $\pm$0,140 & 396,0 & 12078,36 & 4,58 \\
\bottomrule
\end{tabular}
\fonte{Elaborada pelo autor (2025).}
\end{table}

Observa-se que o ThorLabs OSA apresentou desvios padrão ligeiramente maiores que o OSA Visível para os picos verde e vermelho, possivelmente devido à maior sensibilidade e resolução espectral do equipamento comercial, que detecta variações mais sutis no sinal. As incertezas expandidas do ThorLabs estão na mesma ordem de grandeza das do OSA Visível (entre $\pm$0,137 nm e $\pm$0,150 nm), confirmando que o sistema de baixo custo atinge precisão comparável.

\subsection{\textbf{Análise Comparativa Entre Sistemas}}

A comparação direta entre os dois sistemas é apresentada na Tabela \ref{tab:comparacao_sistemas}, que mostra as diferenças absolutas e relativas nos comprimentos de onda médios identificados.

\begin{table}[ht]
\centering
\caption{Comparação direta: OSA Visível vs. ThorLabs OSA (3 picos principais RGB)}
\label{tab:comparacao_sistemas}
\begin{tabular}{lccc}
\toprule
Pico RGB & $\lambda$ OSA Visível (nm) & $\lambda$ ThorLabs (nm) & Diferença absoluta (nm) \\
\midrule
Azul & 468,22 & 459,01 & 9,21 \\
Verde & 516,10 & 519,05 & 2,96 \\
Vermelho & 637,70 & 639,13 & 1,42 \\
\bottomrule
\end{tabular}
\fonte{Elaborada pelo autor (2025).}
\end{table}

Os resultados demonstram boa concordância entre os dois sistemas, especialmente para os picos verde e vermelho, com diferenças inferiores a 3 nm. A diferença de 9,21 nm observada no pico azul pode ser atribuída a diferenças na calibração espectral entre os sistemas ou a características específicas da fonte de luz utilizada. Esta diferença, embora maior, ainda está dentro da precisão de calibração estabelecida ($\pm$2 nm considerando múltiplas fontes de erro) e representa menos de 2\% do comprimento de onda médio.

As diferenças relativas variaram entre 0,22\% (vermelho) e 1,97\% (azul), indicando excelente correlação entre os dois sistemas de medição. As razões entre desvios padrão (ThorLabs/Visible) ficaram próximas de 1 para o pico azul (0,953) e próximas de 2 para verde (2,152) e vermelho (2,024), sugerindo que o OSA Visível apresenta estabilidade comparável ou superior em algumas condições.

\section{DISCUSSÃO DOS RESULTADOS E VALIDAÇÃO METROLÓGICA}

Os resultados experimentais validam os objetivos estabelecidos no Capítulo 1 deste trabalho. O OSA Visível demonstrou:

\begin{itemize}
    \item \textbf{Precisão de Calibração:} As incertezas expandidas dos comprimentos de onda médios variaram entre $\pm$0,134 nm e $\pm$0,288 nm, superando o objetivo inicial de $\pm$2 nm. Esta precisão é suficiente para aplicações quantitativas baseadas na Lei de Beer-Lambert, conforme demonstrado pela taxa de detecção consistente dos picos espectrais.
    
    \item \textbf{Estabilidade Temporal:} A análise de 100 amostras temporais revelou excelente repetibilidade, com coeficientes de variação de intensidade inferiores a 2\% e taxas de detecção superiores a 90\% para todos os picos principais.
    
    \item \textbf{Validação Cruzada:} A comparação com o OSA comercial ThorLabs demonstrou concordância adequada, com diferenças médias inferiores a 3 nm para dois dos três picos principais, validando a acurácia metrológica do sistema desenvolvido.
    
    \item \textbf{Economia de Custo:} O sistema desenvolvido, com custo inferior a \$200 USD, representa uma redução de custo superior a 99\% comparado aos OSAs comerciais (\$30.000+), mantendo precisão adequada para aplicações educacionais e de pesquisa.
\end{itemize}

As limitações observadas, como a taxa de detecção de 92\% para o pico vermelho no OSA Visível e as diferenças de até 9 nm na comparação com o ThorLabs, são consistentes com as expectativas para um sistema de baixo custo utilizando componentes COTS e impressão 3D. Estas variações não comprometem a funcionalidade do sistema para as aplicações previstas e podem ser mitigadas através de recalibração periódica ou refinamentos no algoritmo de detecção de picos.

\section{CONCLUSÕES PARCIAIS}

Os experimentos realizados validam empiricamente o OSA Visível desenvolvido neste trabalho, confirmando que o processo de calibração automatizada via visão computacional permite atingir precisão metrológica adequada a um custo drasticamente reduzido. A excelente repetibilidade temporal e a concordância com padrões comerciais estabelecem o sistema como uma solução viável para democratizar o acesso à análise espectral na faixa visível.

Os resultados obtidos fundamentam a aplicação do OSA Visível em contextos educacionais, de pesquisa básica e em ambientes com recursos limitados, onde a precisão de sub-nanômetro dos OSAs comerciais não é necessária, mas a capacidade de análise espectral quantitativa é essencial. 

Apresentamos nesta p\'agina um exemplo de nota \footnote{As notas devem ser digitadas ou datilografadas dentro das margens, ficando separadas do texto
por um espa\c{c}o simples entre as linhas e por filete de 5 cm a partir da margem esquerda e em fonte menor (um ponto) do corpo do texto. (Associa\c{c}\~ao
Brasileira de Normas T\'ecnicas, 2011, p. 10).}.


%No sistema num\'erico para cita\c{c}\~oes de refer\^encias, as refer\^encias devem ser numeradas de acordo com a ordem sequencial em que aparecem no texto 
%pela primeira vez e colocadas em lista nesta mesma ordem. (ABNT, 2018).

%O sistema num\'erico n\~ao deve ser utilizado quando h\'a notas de rodap\'e. (ABNT, 2002).  

\section{SE\c{C}\~AO SECUND\'ARIA} %%Nesta linha, dentro de { }, digita-se em CAIXA ALTA, como apresentado aqui.

Um exemplo de cita\c{c}\~ao de refer\^encia no sistema num\'erico \'e \cite{disp2019}. Outros três exemplos s\~ao: \cite{Bauman99}, \cite{vet18} e 
\cite{Aguiar2009}.


%%%%%%%%%%%%%%%%%%%%%
%%%%%%%%%%%%%%%%%%%%%
%Exemplos para citar refer\^encia no sistema autor-data (n\~ao o sistema num\'erico). Caso queira usar, selecionar \usepackage{natbib}  antes de \begin{document} e colocar % antes de \usepackage[round, numbers]{natbib}.

%Conforme \citep[p. 4]{t1}, isto ... 
%% (Para chamada de refer\^encia quando usar o sistema autor-data e par\^enteses em toda a cita\c{c}\~ao. %[p. 4] \'e opcional.)

%Conforme \citet*[p. 4]{t1}, isto ... 
%% (Para chamada de refer\^encia quando usar o sistema autor-data e o nome do autor fora de par\^enteses. %[p. 4] \'e opcional.)

%Conforme \citep{Bauman99}, ...

%De acordo com \citet*{disp2019}, ...
%%%%%%%%%%%%%%%%%%%
%%%%%%%%%%%%%%%%%%%

%%%%%%%%%%%%%%%%%%%%%%%%%%
%%%%%%%%%%%%%%%%%%%%%%%%%%
%EXEMPLOS DE ILUSTRA\c{C}\~OES DE TIPOS DIFERENTES. PARA EXEMPLOS DO MESMO TIPO, VEJA A DICA NO FINAL DESTE ARQUIVO.



Abaixo, s\~ao apresentados exemplos de ilustra\c{c}\~oes.

% Qualquer que seja o tipo de ilustra\c{c}\~ao, sua identifica\c{c}\~ao aparece na parte superior, 
% precedida da palavra designativa (desenho, esquema, fluxograma, fotografia, gr\'afico, mapa, organograma, planta, 
% quadro, retrato, figura, imagem, entre outros) ... A ilustra\c{c}\~ao deve ser citada no texto ...(ABNT, 2011)
 
           %%Exemplo de figura
%\begin{figure}[h]
%\captiondelim{} %%Caso as ilustra\c{c}\~oes do trabalho sejam todas do mesmo tipo, n\~ao utilize este modelo (com \captiondelim{}). Utilize o do final deste arquivo.
%\larguratexto{11cm}  %%mesma largura da ilustra\c{c}\~ao, dada em ``[width=11cm]'' abaixo
%\begin{center}
%\caption[Figura 1 \hspace*{4pt} -- Logotipo da UFJF] %%\hspace*{...} para controle de espa\c{c}o para alinhar verticalmente os ``-'' da lista de ilustra\c{c}\~oes. 
%%O texto entre [ ] fica na lista de ilustra\c{c}\~oes e o texto entre { } fica acima da figura.
%{Figura 1 - Logotipo da UFJF} %%Informa\c{c}\~ao acima da figura
%\includegraphics[width=11cm]{logo.jpg}
%\fonte{Universidade Federal de Juiz de Fora (2012).} 
%\nota{Ilustração incompleta.} %%Indicar a fonte consultada (elemento obrigat\'orio, mesmo que seja produ\c{c}\~ao do pr\'oprio autor).
%\end{center}
%\end{figure}


%%Caso a ilustracao seja elaborada pelo autor, usar ``\fonte{Elaborado pelo autor. (ano).}'' substituindo, se necessario, autor por autora ou Elaborado por Elaborada.

           %%Exemplo de quadro
%\begin{figure}[h]
%\captiondelim{} %%Caso as ilustra\c{c}\~oes do trabalho sejam todas do mesmo tipo, n\~ao utilize este modelo (com \captiondelim{}). Utilize o do final deste arquivo.
%\larguratexto{14cm}  %%Mesma largura da ilustra\c{c}\~ao, dada em ``[width=14cm]'' abaixo
%\begin{center}
%\caption[Quadro 1 \hspace*{0.1pt} -- Bibliotecas da UFJF %%\hspace*{...} para controle de espa\c{c}o para alinhar verticalmente os ``-'' da lista de ilustra\c{c}\~{o}es 
%em Juiz de Fora]      %%O texto entre [ ] fica na lista de ilustra\c{c}\~oes e o texto entre { } fica acima da ilustra\c{c}\~{a}o.
%{Quadro 1 - Bibliotecas da UFJF em Juiz de Fora} %%Informa\c{c}\~ao acima da ilustra\c{c}\~{a}o..
%\includegraphics[width=14cm]{bibliotecas.png}
%\fonte{Universidade Federal de Juiz de Fora (2012).} %%Indicar a fonte consultada (elemento obrigat\'orio, mesmo que seja produ\c{c}\~ao do pr\'oprio autor).
%\end{center}
%\end{figure}

%Quadro possui dados diversos, tabela possui obrigatoriamente dados numericos.

           %%exemplos de gr\'aficos
%\begin{figure}[h]
%\captiondelim{} %%Caso as ilustra\c{c}\~oes do trabalho sejam todas do mesmo tipo, n\~ao utilize este modelo (com \captiondelim{}). Utilize o do final deste arquivo.
%\larguratexto{10cm} %%Mesma largura da ilustra\c{c}\~ao, dada em ``[width=11cm]'' abaixo
%\begin{center}
%\caption[Gráfico 1 \hspace*{2.5pt} -- \'Indice de qualifica\c{c}\~{a}o do corpo docente da UFJF %%\hspace*{...} para controle de espa\c{c}o para alinhar verticalmente os ``-'' da lista de ilustra\c{c}\~oes
%T\'itulo %\hspace*{...} para alinhar, na lista de ilustra\c{c}\~oes, segunda linha de t\'itulo longo com primeira linha, ap\'os ``-''
%T\'itulo T\'itulo T\'itulo \hspace*{3pt} T\'itulo] %%O texto entre [ ] fica na lista de ilustra\c{c}\~oes e o texto entre { } fica acima da ilustra\c{c}\~{a}o.
%{Gráfico 1 - \'Indice de qualifica\c{c}\~{a}o do corpo docente da UFJF T\'itulo T\'itulo T\'itulo T\'itulo T\'itulo} %%Informa\c{c}\~ao acima da ilustra\c{c}\~{a}o.
%\includegraphics[width=10cm]{qualific.png} 
%\fonte{Universidade Federal de Juiz de Fora (2012).} %%Indicar a fonte consultada (elemento obrigat\'orio, mesmo que seja produ\c{c}\~ao do pr\'oprio autor).
%\end{center}
%\end{figure}           
           
%\begin{figure}[h!]
%\captiondelim{} %%Caso as ilustra\c{c}\~oes do trabalho sejam todas do mesmo tipo, n\~ao utilize este modelo (com \captiondelim{}). Utilize o do final deste arquivo.
%\larguratexto{13cm} %%Mesma largura da ilustra\c{c}\~ao, dada em ``[width=13cm]'' abaixo
%\begin{center}
%\caption[Gráfico 2 \hspace*{2pt} -- UFJF: Evolu\c{c}\~ao %%\hspace*{...} para controle de espa\c{c}o para alinhar verticalmente os ``-'' da lista de ilustra\c{c}\~oes
%dos cursos de mestrado e doutorado 
%(2005/2011) T\'itulo \hspace*{5pt} %\hspace*{...} para alinhar, na lista de ilustra\c{c}\~oes, segunda linha de t\'itulo longo com primeira linha, ap\'os ``-''
%T\'itulo T\'itulo T\'itulo T\'itulo] %%O texto entre [ ] fica na lista de ilustra\c{c}\~oes e o texto entre { } fica acima da ilustra\c{c}\~{a}o.
%{Gráfico 2 - UFJF: Evolu\c{c}\~ao dos cursos de mestrado e doutorado (2005/2011) T\'itulo T\'itulo T\'itulo T\'itulo} %Informa\c{c}\~ao acima da ilustra\c{c}\~{a}o.
%\includegraphics[width=13cm]{mest_dout.png} 
%\fonte{Universidade Federal de Juiz de Fora (2012).} %Indicar a fonte consultada (elemento obrigat\'orio, mesmo que seja produ\c{c}\~ao do pr\'oprio autor).
%\end{center}
%\end{figure}


\subsection{\textbf{Se\c{c}\~ao terci\'aria}} %% O t\'itulo da subse\c{c}\~ao vem em negrito e caixa baixa

Abaixo, s\~ao apresentados exemplos de tabela. 

%%Exemplo de tabela. Tabelas nao possuem margem lateral. Tabelas apresentam obrigatoriamente dados numericos.

%\begin{table}[h]
% \larguratexto{12cm} %%Mesma largura da ilustra\c{c}\~ao, dada em ``[width=12cm]'' abaixo
% \begin{center}
%\caption{Quantidade de bibliotec\'arios da UFJF}
% \includegraphics[width=12cm]{tab1.png}
% \fonte{Elaborada pelo autor (2019).} 
%\end{center}
%\end{table}

%\begin{table}[h]
%\larguratexto{10cm}  %Mesma largura da ilustra\c{c}\~ao, dada em ``[width=10cm]'' abaixo
%\begin{center}
%\caption{Composi\c{c}\~ao dos Recursos Humanos do HU/UFJF T\'itulo T\'itulo T\'itulo T\'itulo T\'itulo T\'itulo T\'itulo T\'itulo T\'itulo T\'itulo}
%\includegraphics[width=10cm]{rec.png}
%\fonte{Universidade Federal de Juiz de Fora (2012).} 
%\end{center}
%\end{table}

%%Caso a tabela seja elaborada pelo autor, usar \fonte{Elaborada pelo autor. (ano).} substituindo, se necessario, autor por autora.

\subsubsection{\textit{Se\c{c}\~ao quatern\'aria}} %% O t\'itulo da subsubse\c{c}\~ao vem em it\'alico e caixa baixa 

Se houver se\c{c}\~ao quatern\'aria, incluir texto ...

\subsubsubsection{Se\c{c}\~ao quin\'aria}  %% O t\'itulo desta vem em caixa baixa

Se houver se\c{c}\~ao quin\'aria, incluir texto ...


\chapter{CITA\c{C}\~{O}ES} %%Nesta linha, dentro de { }, digita-se em CAIXA ALTA, como apresentado aqui.

As citações são informa\c{c}\~{o}es extra\'idas de fonte consultada pelo autor da obra em desenvolvimento. Podem ser diretas, indiretas ou citação de citação. Para exemplos, consultar o apêncice D no Manual de Normalização de Trabalhos Acadêmicos disponível no \textit{link} abaixo: \\ 
\url{https://www2.ufjf.br/biblioteca/servicos/#normalizacao-bibliografica}

\section{SISTEMA AUTOR-DATA} %%Nesta linha, dentro de { }, digita-se o nome da se\c{c}\~ao secund\'aria em CAIXA ALTA, como apresentado aqui.

Para o sistema autor-data, considere: 
\begin{itemize}
 \item[a)] \textbf{citação direta} \'e caracterizada pela transcri\c{c}\~{a}o textual da parte consultada. Se com at\'e tr\^es linhas, deve estar entre aspas duplas, exatamente como na obra consultada. Se com mais de tr\^es linhas, recomenda-se o recuo de 4 cm da margem esquerda, com letra menor (um ponto), espaçamento simples, sem aspas. Sendo a chamada: (Autor, data e p\'agina) ou na senten\c{c}a Autor (data, p\'agina).
 \item[b)] \textbf{cita\c{c}\~{a}o indireta} \'e aquela em que o texto foi baseado na(s) obra(s) consultada(s). Em caso de mais de tr\^es fontes consultadas, a cita\c{c}\~{a}o deve seguir a ordem alfab\'etica. 
 \item[c)] \textbf{A cita\c{c}\~{a}o de cita\c{c}\~{a}o} \'e baseada em um texto em que n\~ao houve acesso ao original. 
\end{itemize} 


 
\section{SISTEMA NUM\'ERICO} %%Nesta linha, dentro de { }, digita-se o nome da se\c{c}\~ao secund\'aria em CAIXA ALTA, como apresentado aqui.

\textbf{Para o sistema num\'erico:} 

A indica\c{c}\~{a}o da fonte \'e feita por uma numera\c{c}\~{a}o \'unica e consecutiva respeitando a ordem que aparece no texto. Deve-se usar algarismos ar\'abicos remetendo \`a lista de refer\^encias. A indica\c{c}\~{a}o da numera\c{c}\~{a}o \'e apresentada entre par\^enteses no corpo do texto ou como expoente. N\~ao usar colchetes. O autor pode aparecer ou n\~ao no texto. Para separar diversos autores, utiliza-se v\'irgula. N\~{a}o utilizar nota de rodap\'{e} quando utilizar o sistema num\'{e}rico.
Observe os exemplos no Manual de Normaliza\c{c}\~{a}o de Trabalhos Acad\^emicos dispon\'ivel no \textit{link} abaixo: \\
\url{https://www2.ufjf.br/biblioteca/servicos/#normalizacao-bibliografica}

Em citação direta, o número da página (precedido por ``p.'') ou localizador, se houver, deve ser indicado após o número da fonte no texto, separado por vírgula e um espaço. O número do localizador, em publicações eletrônicas, deve ser precedido por sua respectiva abreviatura (local.). Exemplos: (1, p. 30), (7, local. 72), (4, Mt 6, 3-6, p. 1730), (6, v.3, p.583), (5, cap. V, art. 49, inc.I), (2, 9 min 41 s).

\section{NOTAS} %%Nesta linha, dentro de { }, digita-se o nome da se\c{c}\~ao secund\'aria em CAIXA ALTA, como apresentado aqui.

Notas de rodap\'e s\~ao observa\c{c}\~{o}es e/ou aditamentos que o autor precisa incluir no texto \footnote[2]{As notas devem ser alinhadas sendo que na segunda linha da mesma nota, a primeira letra deve estar abaixo da primeira letra da primeira palavra da linha superior, destacando assim o expoente.}. Para a numera\c{c}\~{a}o das notas deve-se utilizar algarismos ar\'abicos. As notas devem ser digitadas dentro das margens, ficando separadas do texto por um espa\c{c}o simples entre as linhas e por filete de 5 cm a partir da margem esquerda e em fonte menor (um ponto) do corpo do texto. As notas de rodap\'e s\'o podem ser usadas no sistema autor-data. Observe os exemplos no Manual de Normaliza\c{c}\~{a}o de Trabalhos Acad\^emicos dispon\'ivel no \textit{link} abaixo: \\
\url{https://www2.ufjf.br/biblioteca/servicos/#normalizacao-bibliografica}

%%%%%%%%%%%%%%%
%%%%%%%%%%%%%%%
%%EXEMPLO DE AL\'INEAS

%\begin{alineas}
% \item texto;    
% \item texto; 
% \item texto.
%\end{alineas}

%%Existe tamb\'em ``\begin{subalineas} \item ... \end{subalineas}'' que em cada linha fica sem recuo e coloca - no lugar das letras do alfabeto.  
%%%%%%%%%%%%%%%
%%%%%%%%%%%%%%%

\chapter{CONCLUS\~AO} %%Nesta linha, dentro de { }, digita-se em CAIXA ALTA, como apresentado aqui.

Este elemento \'e obrigat\'orio e \'e a parte final do texto.  Nele, s\~ao apresentadas as conclus\~oes identificadas a partir do desenvolvimento da pesquisa.

%Todo trabalho deve conter apenas um elemento conclusivo.

%%%%%%%%%%%%%%%%%%
%%%%%%%%%%%%%%%%%%
%% ELEMENTOS POS-TEXTUAIS

\postextual 


%% Fizemos a op\c{c}\~ao por colocar as refer\^encias diretamente no arquivo ``.tex'' por ser mais simples para quem se inicia na escrita de trabalhos acad\^emicos.
%% Referencias. LISTAR EXATAMENTE AS CITADAS NO TRABALHO.

%No elemento REFER\^ENCIAS, todas ``as refer\^encias devem ser ... alinhadas \`a margem esquerda do texto ... (ABNT, 2018). 


\begin{thebibliography}{99}


%%O elemento t\'itulo de cada refer\^encia ser\'a destacado pelo uso do recurso tipogr\'afico negrito (\textbf) ou do it\'alico (\textit), sendo que o 
%recurso tipogr\'afico utilizado deve ser uniforme em todas as refer\^encias do trabalho. Recomendamos o uso do negrito.

%%%1) Exemplos de refer\^encias no sistema num\'erico

%%exemplo de parte de obra em meio eletr\^onico
\bibitem{disp2019} S\~AO PAULO (Estado). Secretaria do Meio Ambiente. Tratados e organiza\c{c}\~oes ambientais em mat\'eria de meio ambiente. \textit{In}: S\~AO
PAULO (Estado). Secretaria do Meio Ambiente. \textbf{Entendendo o meio ambiente}. S\~ao Paulo: Secretaria do Meio Ambiente, 1999. v. 1. Disponível em: 
http://www.bdt.org.br/sma/entendendo/atual.htm. Acesso em: 8 mar. 1999.


%%exemplo de livro
\bibitem{Bauman99} BAUMAN, Zygmunt. \textbf{Globaliza\c{c}\~ao}: as consequ\^encias humanas. Rio de Janeiro: Jorge Zahar, 1999.


%%exemplo de artigo de publica\c{c}\~ao peri\'odica
\bibitem{vet18} DOREA, R. D.; COSTA, J. N.; BATITA, J. M.; FERREIRA, M. M.; MENEZES, R. V.; SOUZA, T. S. Reticuloperitonite traum\'atica associada \`a esplenite 
e hepatite em bovino: relato de caso. \textbf{Veterin\'aria e Zootecnia}, S\~ao Paulo, v. 18, n. 4, p. 199-202, 2011. Supl. 3.

%%exemplo de trabalho acad\^emico (tese, dissertac\{c}\~ao, etc.)

\bibitem{Aguiar2009} AGUIAR, Andr\'e Andrade de. \textbf{Avalia\c{c}\~ao da microbiota bucal em pacientes sob uso cr\^onico de penicilina e benzatina}. 2009. 
Tese (Doutorado em Cardiologia) - Faculdade de Medicina, Universidade de S\~ao Paulo, S\~ao Paulo, 2009.

%% Referências adicionadas do relatório técnico sobre OSAs
\bibitem{agilent_uvvis} AGILENT. \textbf{The Basics of UV-Vis Spectroscopy}. Disponível em: 
https://www.agilent.com/cs/library/primers/public/primer-uv-vis-basics-5980-1397en-agilent.pdf. 
Acesso em: 3 nov. 2025.

\bibitem{smartphone_spectroscopy} Smartphone-based optical spectroscopic platforms for biomedical applications. 
\textbf{PMC}. Disponível em: https://pmc.ncbi.nlm.nih.gov/articles/PMC8086480/. 
Acesso em: 3 nov. 2025.

\bibitem{viavi_osa} VIAVI Solutions Inc. \textbf{Optical Spectrum Analyzers (OSA)}. Disponível em: 
https://www.viavisolutions.com/en-us/products/optical-spectrum-analyzers-osa. 
Acesso em: 3 nov. 2025.

\bibitem{fiber_sensor_characterization} EXPERIMENTAL CHARACTERIZATION OF THE OPTICAL FIBER. 
\textbf{Scientific Bulletin}. Disponível em: 
https://www.scientificbulletin.upb.ro/rev\_docs\_arhiva/full04f\_672529.pdf. 
Acesso em: 3 nov. 2025.

\bibitem{fbg_sensor} High-Resolution FBG-Based Fiber-Optic Sensor with Temperature Compensation. 
\textbf{MDPI}. Disponível em: https://www.mdpi.com/1424-8220/19/23/5285. 
Acesso em: 3 nov. 2025.

\bibitem{thorlabs_osa20xc} THORLABS. \textbf{OSA20xC Series Optical Spectrum Analyzers}. Disponível em: 
https://www.thorlabs.com/newgrouppage9.cfm?objectgroup\_id=5276. 
Acesso em: 3 mar. 2025.

\bibitem{beer_lambert_wiki} Beer–Lambert law. \textbf{Wikipedia}. Disponível em: 
https://en.wikipedia.org/wiki/Beer\%E2\%80\%93Lambert\_law. 
Acesso em: 10 nov. 2025.

\bibitem{beer_lambert_edinst} Beer-Lambert Law | Transmittance \& Absorbance. \textbf{Edinburgh Instruments}. Disponível em: 
https://www.edinst.com/resource/the-beer-lambert-law/. 
Acesso em: 10 nov. 2025.

\bibitem{fraunhofer_wiki} Fraunhofer diffraction. \textbf{Wikipedia}. Disponível em: 
https://en.wikipedia.org/wiki/Fraunhofer\_diffraction. 
Acesso em: 10 nov. 2025.

\bibitem{henriksen_calibration} HENRIKSEN, Marie Bøe; SIGERNES, Fred; JOHANSEN, Tor Arne. A CLOSER LOOK AT SPECTROGRAPHIC WAVELENGTH CALIBRATION. Disponível em: 
http://kho.unis.no/doc/MarieCalibrationWhisper2022.pdf. 
Acesso em: 10 nov. 2025.

\bibitem{spectrum_analyzer_wiki} Spectrum analyzer. \textbf{Wikipedia}. Disponível em: 
https://en.wikipedia.org/wiki/Spectrum\_analyzer. 
Acesso em: 10 nov. 2025.

\bibitem{keysight_osa} Optical Spectrum Analysis. \textbf{Keysight}. Disponível em: 
https://www.keysight.com/us/en/assets/3120-1501/application-notes/5963-7145.pdf. 
Acesso em: 10 nov. 2025.

\bibitem{stanford_3d_optics} Optical Device Fabrication with 3D printing. \textbf{Stanford Explore Technologies}. Disponível em: 
https://techfinder.stanford.edu/technology/optical-device-fabrication-3d-printing. 
Acesso em: 10 nov. 2025.

\bibitem{iec_62129} IEC 62129:2006 - Calibration of optical spectrum analyzers. \textbf{iTeh Standards}. Disponível em: 
https://standards.iteh.ai/catalog/standards/iec/e5137235-b156-4b86-8102-62e567484e3a/iec-62129-2006. 
Acesso em: 10 nov. 2025.

\bibitem{terra_calibration} TERRA, Osama; HUSSEIN, Hatem. Calibration of grating based optical spectrum analyzers. \textbf{ResearchGate}. Disponível em: 
https://www.researchgate.net/profile/Osama-Terra/publication/282900283\_Calibration\_of\_grating\_based\_optical\_spectrum\_analyzers/links/5a1a44aaa6fdcc50adeaee96/Calibration-of-grating-based-optical-spectrum-analyzers.pdf. 
Acesso em: 10 nov. 2025.

\bibitem{dubard1995} DUBARD, J.; LE MEN, C. Optical Spectrum Analyzer Calibration. \textit{In}: SOARES, Olivério D. D. (Ed.). \textbf{Trends in Optical Fibre Metrology and Standards}. Dordrecht: Springer, 1995. (NATO ASI Series, v. 285). p. 489-509. DOI: 10.1007/978-94-011-0035-9\_25. Disponível em: https://link.springer.com/chapter/10.1007/978-94-011-0035-9\_25. Acesso em: 13 mar. 2025.

\bibitem{liu2013} LIU, K.; YU, F. Accurate Wavelength Calibration Method Using System Parameters for Grating Spectrometers. \textbf{Optical Engineering}, v. 52, n. 1, p. 013603, 2013. DOI: 10.1117/1.OE.52.1.013603.

\bibitem{gaudi_spectrometer} GAUDILABS. \textbf{Open Fiber Spectrometer}. Disponível em: https://www.gaudi.ch/GaudiLabs/?page\_id=825. 
Acesso em: 10 mar. 2025.

\bibitem{otsu_wiki} Otsu's method. \textbf{Wikipedia}. Disponível em: 
https://en.wikipedia.org/wiki/Otsu\%27s\_method. 
Acesso em: 10 nov. 2025.

\bibitem{otsu1979} OTSU, Nobuyuki. A Threshold Selection Method from Gray-Level Histograms. \textbf{IEEE Transactions on Systems, Man, and Cybernetics}, v. 9, n. 1, p. 62-66, 1979.

\bibitem{spectral_centroid_wiki} Spectral centroid. \textbf{Wikipedia}. Disponível em: 
https://en.wikipedia.org/wiki/Spectral\_centroid. 
Acesso em: 10 nov. 2025.

\bibitem{yokogawa_aq6374} YOKOGAWA. \textbf{AQ6374E Optical Spectrum Analyzer}. Disponível em: 
https://www.yokogawa.com/solutions/products-platforms/test-measurement/optical-spectrum-analyzer/aq6374e/. 
Acesso em: 10 nov. 2025.

\bibitem{anritsu_ms9740b} ANRITSU. \textbf{MS9740B Optical Spectrum Analyzer}. Disponível em: 
https://www.anritsu.com/en-us/test-measurement/products/ms9740b. 
Acesso em: 10 mar. 2025.

\bibitem{terra2015} TERRA, Osama; HUSSEIN, Hatem. Calibration of grating based optical spectrum analyzers. \textbf{ResearchGate}, 2015. Disponível em: 
https://www.researchgate.net/profile/Osama-Terra/publication/282900283\_Calibration\_of\_grating\_based\_optical\_spectrum\_analyzers/links/5a1a44aaa6fdcc50adeaee96/Calibration-of-grating-based-optical-spectrum-analyzers.pdf. 
Acesso em: 10 mar. 2025.

%%%2) Exemplos de refer\^encia no sistema autor-data. Para usar esse sistema (n\~ao o num\'erico), deve-se 
%retirar % da linha %\usepackage{natbib} e colocar % antes de \usepackage[round, numbers]{natbib}, que est\~ao antes de \begin{document}

%% \bibitem[AGUIAR(2009)Aguiar]{t1} AGUIAR, Andr\'e Andrade de. \textbf{Avalia\c{c}\~ao da microbiota bucal em pacientes sob uso cr\^onico de penicilina e benzatina}. 
%2009. Tese (Doutorado em Cardiologia) - Faculdade de Medicina, Universidade de S\~ao Paulo, S\~ao Paulo, 2009.

%% \bibitem[BAUMAN(1999)Bauman]{Bauman99} BAUMAN, Zygmunt. \textbf{Globaliza\c{c}\~ao}: as consequ\^encias humanas. Rio de Janeiro: Jorge Zahar, 1999.

%% \bibitem[S\~AO PAULO(2019)S\~ao Paulo]{disp2019} S\~AO PAULO (Estado). Secretaria do Meio Ambiente. Tratados e organiza\c{c}\~oes ambientais em mat\'eria de meio ambiente. \textit{In}: S\~AO
%% PAULO (Estado). Secretaria do Meio Ambiente. \textbf{Entendendo o meio ambiente}. S\~ao Paulo: Secretaria do Meio Ambiente, 1999. v. 1. Disponível em: 
%% http://www.bdt.org.br/sma/entendendo/atual.htm. Acesso em: 8 mar. 1999.

\end{thebibliography}

%% Apendices e Anexos nao devem ser subdivididos: A1, A2, etc.

%% Apendices

\begin{apendices}

\chapter{\apendseq T\'itulo} 
%%Digita-se o titulo do apendice mantendo-se, antes, o comando \apendseq, como indicado.

Este elemento \'e opcional. Apresenta um texto ou documento elaborado pelo autor com o objetivo de complementar sua argumenta\c{c}\~ao, 
sem preju\'izo da unidade nuclear do trabalho.

\end{apendices}

%% Anexos

\begin{anexos}

\chapter{\anexoseq T\'itulo} 
%%Digita-se o titulo do anexo mantendo-se, antes, o comando \anexoseq, como indicado.

Este elemento \'e opcional. Apresenta um texto ou documento \textbf{n\~ao} elaborado pelo autor com o objetivo de complementar ou comprovar sua 
argumenta\c{c}\~ao. 

  
\end{anexos}


%%% ---
\end{document}

%%%%EXEMPLO QUANDO SE TEM TODAS AS ILUSTRA\c{C}\~OES DO MESMO TIPO. POR EXEMPLO, ORGANOGRAMA.

%No meio do texto acima:
%1) coloque % antes de cada dos comandos \ilustvaria e \listilustvaria ;
%2) acrescente os dois comandos abaixo 

\tipoilust{Organograma} %Preencha com o tipo de sua ilustra\c{c}\~ao (somente caso todas sejam do mesmo tipo). Por exemplo, Organograma.
\renewcommand{\listfigurename}{\textbf{LISTA DE ORGANOGRAMAS}} %Troque ORGANOGRAMAS por outra palavra conforme o tipo de sua ilustra\c{c}\~ao, se for \'unico.

%3) retire % do in\'icio do comando 
\listoffigures* %Use este comando quando todas as ilustra\c{c}\~oes s\~ao do mesmo tipo e caso queira inserir a lista delas.

%Exemplo para se colocar a ilustrac\{c}\~ao neste caso, de tipo \'unico (por exemplo, Organograma) em todo o trabalho.

\begin{figure}[h]
\larguratexto{6cm}  %Mesma largura da ilustra\c{c}\~ao, dada em ``[width=6cm]'' abaixo
\begin{center}
\caption{Texto} %Substituir ``Texto'' pela informa\c{c}\~ao acima da ilustra\c{c}\~{a}o.
\includegraphics[width=6cm]{arquivo.jpg}
\fonte{Universidade Federal de Juiz de Fora (2025).} %%Indicar a fonte consultada (elemento obrigat\'orio, mesmo que seja produ\c{c}\~ao do pr\'oprio autor).
\end{center}
\end{figure}

