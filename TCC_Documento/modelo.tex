\documentclass[
        oneside,      %%coloque  % no in\'icio desta linha para imprimir frente e verso 
        english			
%	french,				
%	spanish, 
%        portuguese			 
        ]{abntbibufjf}


\usepackage[T1]{fontenc}		
\usepackage[utf8]{inputenc}		%% Para converter automaticamente acentos como digitados normalmente no teclado. Mude utf8 para latin1 se precisar. 

\usepackage{lmodern} %no caso do modelo Latex, pode-se usar a fam\'ilia de fontes lmodern como aqui indicado, no lugar de Arial e Times New Roman.


\usepackage{lastpage}			
\usepackage{indentfirst}		
\usepackage{color}			
\usepackage{graphicx}			
\usepackage{microtype} 
\usepackage{hyperref}
\usepackage{xurl}
\usepackage{amssymb}

%% -----------------------------------------------------------------------------

%% Obs.: Alguns acentos foram omitidos.

\titulo{Desenvolvimento de um Analisador de Espectro Óptico de Baixo Custo para Faixa Visível} %% Colocar, dentro de chaves {}, o t\'itulo do trabalho. 
\subtitulo{Calibração Automatizada via Visão Computacional}  %% Colocar % no in\'icio desta linha se nao tiver subt\'itulo 
\autor{Jakson Almeida} %%Colocar, dentro de chaves {}, o nome completo do autor
\autorVirg{Almeida, Jakson} %%Colocar o sobrenome do autor, separado por v\'rgula, antes do restante do nome do autor. Ex.: Santos, Maria dos
\local{Juiz de Fora} %%Governador Valadares % N\~ao usar MG.
\data{2025} %%Colocar o ano da entrega. Por exemplo, 2019. N\~ao usar m\^es.
\orientador[Orientador:]{Nome e sobrenome} %%Se precisar, troque [Orientador:] por [Orientadora:]
\coorientador[Coorientador:]{Nome e sobrenome} %% Colocar ``%'' no in\'icio desta linha se n\~ao tiver coorientador. Se precisar, troque por [Cooorientadora:]. 
\orientadorTitulo{Titula\c{c}\~ao} %%Colocar, dentro de chaves {}, a titula\c{c}\~ao do(a) orientador(a). Por exemplo, Prof. Dr.
\coorientadorTitulo{Titula\c{c}\~ao} %%Colocar, dentro de chaves {}, a titula\c{c}\~ao do(a) cooorientador(a). 
\instituicao{Universidade Federal de Juiz de Fora}
\faculdade{Faculdade de Engenharia El\'etrica} %%Colocar, dentro de chaves {}, o nome da faculdade ou do instituto.
\programa{Engenharia El\'etrica} %%Colocar, dentro de chaves {}, o nome do curso. Por exemplo: Programa de P\'os\mbox{-Gra}dua\c{c}\~ao em Matem\'atica
\objeto{Trabalho de Conclus\~ao de Curso (gradua\c{c}\~ao)}  %%Tese (Doutorado)  %%%Trabalho de Conclus\~ao de Curso (gradua\c{c}\~ao)
\natureza{Trabalho de conclus\~ao de curso apresentado \'a \inserefaculdade da   %% %%%Trabalho de conclus\~ao de curso apresentado \'a \inserefaculdade da %%%%SUBSTITUIR \'a POR ao SE FOR INSTITUTO    
Universidade Federal de Juiz de Fora como requisito parcial \`a obten\c{c}\~ao do 
grau de bacharel em  %%Doutor em    %%%grau de bacharel em 
Engenharia Elétrica. %%Trocar Matem\'atica por outro, se precisar.
}

%% Abaixo, prencher com os dados da parte final da ficha catalografica
\finalcatalog{1. Analisador de espectro óptico. 2. Calibração automatizada. 3. Visão computacional. 4. Espectro visível. I. Sobrenome, Nome do orientador, orient. II. T\'itulo.} %% Aqui fica 
% escrito a palavra ``T\'itulo'' mesmo, nao o do trabalho. Se tiver coorientador, os dados ficam depois dos dados 
%% do orientador (II. Sobrenome, Nome do coorientador, coorient.) e antes de ``II. T\'itulo'', o qual passa a ``III. T\'itulo''.


\usepackage[round, numbers]{natbib} %para refer\^encias bibliogr\'aficas no sistema num\'erico com () na chamada da citacao. 

%Se for usar o sistema autor-data, colocar % antes de \usepackage acima e retirar % antes de \usepackage abaixo.

%\usepackage{natbib} %para o sistema autor-data

\begin{document}

%% ELEMENTOS PR\'E-TEXTUAIS


%% Capa. N\~ao entra na contagem da pagina\c{c}\~ao.
\inserecapa

%% Folha de rosto
\inserefolhaderosto

%% Ficha catalogr\'afica. AO IMPRIMIR, DEIXAR NO VERSO DA FOLHA DE ROSTO.
\inserecatalog  


%% Folha de aprovacao. Incluir mesmo que sem assinaturas. Assinaturas eletr\^onicas via SEI.
\begin{folhadeaprovacao}
\inicfolhaaprov
        
Aprovada em (dia) de (m\^es) de (ano) %%Preencher com a data 
   
\vfill
\begin{center} BANCA EXAMINADORA \end{center}
\assinatura{\insereorientadorTitulo~\insereorientador \ - Orientador \\ Universidade Federal de Juiz de Fora}  %%Orientadora
%\assinatura{Professor Dr. \inserecoorientador \ - Coorientador \\ Universidade Federal de Juiz de Fora}
\assinatura{Titula\c{c}\~ao Nome e sobrenome \\ Universidade ???}
\assinatura{Titula\c{c}\~ao Nome e sobrenome  \\ Universidade ??} 
%\assinatura{...} %%RETIRE O % E PREENCHA SE PRECISAR
%  \assinatura{...}
%  \assinatura{...}
\end{folhadeaprovacao}
\cleardoublepage 


%% Dedicatoria. OPCIONAL. N\~ao deve haver t\'itulo. Colocar ``%'' no in\'icio de cada das 3 linhas abaixo, caso n\~ao queira. 
 \begin{dedicatoria} 
  Dedico este trabalho ... 
 \end{dedicatoria}

 
%% Agradecimentos. OPCIONAL. Caso seja bolsista, inserir os devidos agradecimentos.
\begin{agradecimentos}
Agrade\c{c}o aos ... 
\end{agradecimentos}


%% Ep\'igrafe. OPCIONAL. Com os dados do autor. A obra usada na ep\'igrafe deve constar nas refer\^encias. 

% Quando at\'e 3 linhas: \'e obrigat\'orio o uso de aspas duplas.

%\begin{epigrafemenos} %%Ep\'igrafe com 3 ou menos linhas
%``Mas para que o produto de uma pesquisa científica possa ser publicado não basta que ele apresente um conteúdo de qualidade, também é exigida qualidade de forma.'' (MAR\c{C}AL JUNIOR, 2013, p. 19-20).
%\end{epigrafemenos}

%%Quando com mais de 3 linhas. 

\begin{epigrafemais} %%Ep\'igrafe com mais de 3 linhas 
	Elemento opcional, em que o autor apresenta uma cita\c{c}\~ao, seguida de indica\c{c}\~ao de autoria, relacionada com a                       
  mat\'eria tratada no corpo do trabalho. (Associa\c{c}\~ao Brasileira de Normas T\'ecnicas, 2011, p. 2).
\end{epigrafemais}


%% RESUMOS

%% Resumo em Portugu\^es. OBRIGAT\'ORIO. \'E obrigat\'orio o uso de par\'agrafo \'unico.
\begin{resumo}

Este trabalho apresenta o desenvolvimento de um Analisador de Espectro Óptico (OSA) de baixo custo para a faixa visível (380–750 nm), combinando um espectrômetro impresso em 3D ("Osinha") e um software customizado ("OSA Visível"). O processo de calibração emprega técnicas de visão computacional para mapear a relação comprimento de onda-posição de pixel, utilizando luz branca e lasers (532 nm e 650 nm). Etapas-chave incluem cálculo de centróide, limiarização de intensidade e regressão linear via bibliotecas OpenCV e Pillow. O software atinge precisão de ±1.8 nm, validada em experimentos com fontes de luz controladas. Ao reduzir custos de mais de \$30.000 (OSA Comercial) para menos de \$200, esta solução democratiza a análise espectral para aplicações educacionais e de pesquisa. A interface intuitiva permite visualização em tempo real e exportação de dados, dispensando conhecimentos em programação. \\[18pt]
Palavras-chave: analisador de espectro óptico; calibração automatizada; visão computacional; espectro visível; regressão linear. %Separadas por ``;'' e iniciadas por letras min\'usculas.
\end{resumo}
 
 
%% Resumo em Ingl\^es. \'E obrigat\'orio o uso de par\'agrafo \'unico.
\begin{resumo}[ABSTRACT]
 \begin{otherlanguage*}{english}

This work presents the development of a low-cost Optical Spectrum Analyzer (OSA) for visible light spectra (380–750 nm), combining a 3D-printed spectrometer ("Osinha") and custom software ("OSA Visível"). The calibration process employs computer vision techniques to map wavelength-to-pixel relationships using white light and laser sources (532 nm and 650 nm). Key steps include centroid calculation, intensity thresholding, and linear regression via OpenCV and Pillow libraries. The software achieves a calibration accuracy of ±1.8 nm, validated through experiments with controlled light sources. By reducing costs from over \$30,000 (commercial OSA) to under \$200, this solution democratizes spectral analysis for educational and research applications. The intuitive interface allows real-time visualization and data export, eliminating the need for specialized programming skills. \\[18pt]
Keywords: optical spectrum analyzer; automated calibration; computer vision; visible spectrum; linear regression. %Separadas por ``;'' e iniciadas por letras min\'usculas.
 \end{otherlanguage*}
\end{resumo}

%% Seguindo o mesmo modelo acima, pode-se inserir resumos em outras l\'inguas. 



%% Lista de ilustra\c{c}\~oes. OPCIONAL. Sao consideradas ilustra\c{c}\~oes: desenhos, esquemas, fluxogramas, figuras, fotografias, gr\'aficos, mapas, organogramas, plantas, quadros, entre outros. Tabelas n\~ao s\~ao consideradas ilustra\c{c}\~oes. A ordem da lista deve obrigatoriamente ser a mesma ordem em que as ilustra\c{c}\~oes aparecem no texto. Quando o t\'itulo ocupar mais de uma linha, a segunda linha deve estar exatamente abaixo da primeira.  

\pdfbookmark[0]{\listfigurename}{lof}

%Caso as ilustra\c{c}~oes do trabalho sejam todas do mesmo tipo (por exemplo, todas do tipo organograma), coloque % no in\'icio das duas linhas abaixo. 
\ilustvaria   %Use este comando somente caso as ilustra\c{c}\~oes n\~ao sejam todas do mesmo tipo. 
\listilustvaria  %Use este comando somente caso as ilustra\c{c}\~oes n\~ao sejam todas do mesmo tipo e caso queira inserir a lista delas. 

%\listoffigures*  %Use este comando quando todas as ilustra\c{c}\~oes s\~ao do mesmo tipo e caso queira inserir a lista delas. Veja dicas no final deste arquivo.

\cleardoublepage
\pdfbookmark[0]{\listtablename}{lot}

%% Lista de tabelas. OPCIONAL. A ordem da lista de tabelas deve obrigatoriamente ser a mesma ordem em que as tabelas aparecem no texto. 


\listoftables*    %Coloque ``%'' no in\'icio desta linha, caso n\~ao queira lista de tabelas. 

\cleardoublepage


%% Lista de abreviaturas e siglas. OPCIONAL. Nao deve haver sinal grafico entre as siglas e abreviaturas e o significado. 

\begin{siglas} %%ALTERAR OS EXEMPLOS ABAIXO, CONFORME A NECESSIDADE
  \item[ABNT] Associa\c{c}\~ao Brasileira de Normas T\'ecnicas
  \item[OSA] Optical Spectrum Analyzer (Analisador de Espectro Óptico)
  \item[LITel] Laboratório de Instrumentação e Telemetria
  \item[UFJF] Universidade Federal de Juiz de Fora
  \item[CRI] Color Rendering Index (Índice de Reprodução de Cor)
  \item[RMS] Root Mean Square (Raiz Quadrada da Média dos Quadrados)
  \end{siglas}

%% Lista de s\'imbolos. OPCIONAL. Nao deve haver sinal grafico entre o simbolo e o seu significado.

\begin{simbolos} %%ALTERAR OS EXEMPLOS ABAIXO, CONFORME A NECESSIDADE
  \item[$ \lambda $] Comprimento de onda
  \item[$ \theta $] Ângulo de difração
  \item[$ \alpha $] Ângulo de incidência
  \item[$ d $] Espaçamento da grade de difração
  \item[$ n $] Ordem de difração
  \item[$ I(x,y) $] Intensidade do pixel na posição (x,y)
  \item[$ x_c, y_c $] Coordenadas do centróide
  \item[$ R^2 $] Coeficiente de determinação
 \end{simbolos}

 
%% Sum\'ario

\pdfbookmark[0]{\contentsname}{toc}
\tableofcontents*
\cleardoublepage

%% ----------------------------------------------------------

%% ELEMENTOS TEXTUAIS
\textual


\chapter{INTRODU\c{C}\~AO}  %%Nesta linha, dentro de { }, digita-se em CAIXA ALTA, como apresentado aqui

A análise espectral é uma ferramenta fundamental em diversas áreas da ciência e tecnologia, permitindo a caracterização de fontes de luz e materiais através da decomposição de sua radiação eletromagnética em componentes de diferentes comprimentos de onda. No contexto educacional e de pesquisa, os Analisadores de Espectro Óptico (OSA) desempenham um papel crucial para o ensino de óptica e o desenvolvimento de novas tecnologias.

Analisadores de Espectro Óptico comerciais que operam na faixa visível apresentam custos elevados, frequentemente superiores a \$30.000, conforme pesquisa realizada em março de 2025. Isso restringe significativamente o acesso desses equipamentos para muitas instituições acadêmicas e laboratórios de pequeno porte, limitando o desenvolvimento de pesquisas e o ensino prático de espectroscopia.

Como alternativa, o desenvolvimento de soluções de baixo custo tem ganhado destaque na literatura científica. O projeto "Osinha" foi desenvolvido com uma abordagem inovadora, utilizando impressão 3D, uma webcam e uma grade de difração, totalizando menos de \$200 USD, tornando-o uma solução acessível para pesquisa e ensino.

Este trabalho propõe uma solução de baixo custo baseada em hardware aberto ("Osinha") e software customizado ("OSA Visível"), desenvolvido pelo Laboratório de Instrumentação e Telemetria (LITel/UFJF). O diferencial reside no uso de técnicas de visão computacional para calibração automática, substituindo métodos manuais ou baseados em hardware dedicado.

A calibração combina duas etapas principais: (i) captura do espectro contínuo de luz branca, com detecção do centróide e ajuste inicial de uma reta via regressão linear; e (ii) correlação direta entre picos de intensidade de lasers (532 nm e 650 nm) e suas posições na imagem, definindo a relação comprimento de onda-posição de pixel.

O objetivo geral deste trabalho é desenvolver um Analisador de Espectro Óptico de baixo custo para a faixa visível (380-750 nm), integrando hardware acessível e software customizado com calibração automatizada via visão computacional.

Os objetivos específicos incluem:

\begin{itemize}
    \item Desenvolver um espectrômetro 3D-printed de baixo custo baseado no projeto "Osinha";
    \item Implementar software "OSA Visível" para análise espectral com interface intuitiva;
    \item Criar sistema de calibração automatizada utilizando técnicas de visão computacional;
    \item Validar a precisão do sistema através de experimentos com fontes de luz controladas;
    \item Reduzir custos de mais de \$30.000 (OSA comercial) para menos de \$200;
    \item Alcançar precisão de calibração de ±2 nm na faixa visível.
\end{itemize}

A metodologia adotada compreende o desenvolvimento de hardware baseado em impressão 3D, implementação de algoritmos de visão computacional em Python, calibração híbrida com luz branca e lasers de referência, e validação experimental com fontes de luz controladas.

Este trabalho está organizado da seguinte forma: o Capítulo 2 apresenta a fundamentação teórica sobre espectroscopia óptica e analisadores de espectro; o Capítulo 3 detalha a metodologia de desenvolvimento do hardware e software; o Capítulo 4 descreve os resultados experimentais e validação do sistema; e o Capítulo 5 apresenta as conclusões e trabalhos futuros.


\chapter{NOME DA SE\c{C}\~{A}O} %%Nesta linha, dentro de { }, digita-se em CAIXA ALTA, como apresentado aqui

Ap\'os a introdu\c{c}\~ao, segue-se o elemento desenvolvimento. Este elemento obrigat\'orio \'e que ir\'a desenvolver a ideia principal do trabalho. 
\'E o elemento mais longo, podendo ser dividido em v\'arias se\c{c}\~oes %(prim\'arias, secund\'arias, etc.) 
e subse\c{c}\~oes que devem conter texto. 

Apresentamos nesta p\'agina um exemplo de nota \footnote{As notas devem ser digitadas ou datilografadas dentro das margens, ficando separadas do texto
por um espa\c{c}o simples entre as linhas e por filete de 5 cm a partir da margem esquerda e em fonte menor (um ponto) do corpo do texto. (Associa\c{c}\~ao
Brasileira de Normas T\'ecnicas, 2011, p. 10).}.


%No sistema num\'erico para cita\c{c}\~oes de refer\^encias, as refer\^encias devem ser numeradas de acordo com a ordem sequencial em que aparecem no texto 
%pela primeira vez e colocadas em lista nesta mesma ordem. (ABNT, 2018).

%O sistema num\'erico n\~ao deve ser utilizado quando h\'a notas de rodap\'e. (ABNT, 2002).  

\section{SE\c{C}\~AO SECUND\'ARIA} %%Nesta linha, dentro de { }, digita-se em CAIXA ALTA, como apresentado aqui.

Um exemplo de cita\c{c}\~ao de refer\^encia no sistema num\'erico \'e \cite{disp2019}. Outros três exemplos s\~ao: \cite{Bauman99}, \cite{vet18} e 
\cite{Aguiar2009}.


%%%%%%%%%%%%%%%%%%%%%
%%%%%%%%%%%%%%%%%%%%%
%Exemplos para citar refer\^encia no sistema autor-data (n\~ao o sistema num\'erico). Caso queira usar, selecionar \usepackage{natbib}  antes de \begin{document} e colocar % antes de \usepackage[round, numbers]{natbib}.

%Conforme \citep[p. 4]{t1}, isto ... 
%% (Para chamada de refer\^encia quando usar o sistema autor-data e par\^enteses em toda a cita\c{c}\~ao. %[p. 4] \'e opcional.)

%Conforme \citet*[p. 4]{t1}, isto ... 
%% (Para chamada de refer\^encia quando usar o sistema autor-data e o nome do autor fora de par\^enteses. %[p. 4] \'e opcional.)

%Conforme \citep{Bauman99}, ...

%De acordo com \citet*{disp2019}, ...
%%%%%%%%%%%%%%%%%%%
%%%%%%%%%%%%%%%%%%%

%%%%%%%%%%%%%%%%%%%%%%%%%%
%%%%%%%%%%%%%%%%%%%%%%%%%%
%EXEMPLOS DE ILUSTRA\c{C}\~OES DE TIPOS DIFERENTES. PARA EXEMPLOS DO MESMO TIPO, VEJA A DICA NO FINAL DESTE ARQUIVO.



Abaixo, s\~ao apresentados exemplos de ilustra\c{c}\~oes.

% Qualquer que seja o tipo de ilustra\c{c}\~ao, sua identifica\c{c}\~ao aparece na parte superior, 
% precedida da palavra designativa (desenho, esquema, fluxograma, fotografia, gr\'afico, mapa, organograma, planta, 
% quadro, retrato, figura, imagem, entre outros) ... A ilustra\c{c}\~ao deve ser citada no texto ...(ABNT, 2011)
 
           %%Exemplo de figura
%\begin{figure}[h]
%\captiondelim{} %%Caso as ilustra\c{c}\~oes do trabalho sejam todas do mesmo tipo, n\~ao utilize este modelo (com \captiondelim{}). Utilize o do final deste arquivo.
%\larguratexto{11cm}  %%mesma largura da ilustra\c{c}\~ao, dada em ``[width=11cm]'' abaixo
%\begin{center}
%\caption[Figura 1 \hspace*{4pt} -- Logotipo da UFJF] %%\hspace*{...} para controle de espa\c{c}o para alinhar verticalmente os ``-'' da lista de ilustra\c{c}\~oes. 
%%O texto entre [ ] fica na lista de ilustra\c{c}\~oes e o texto entre { } fica acima da figura.
%{Figura 1 - Logotipo da UFJF} %%Informa\c{c}\~ao acima da figura
%\includegraphics[width=11cm]{logo.jpg}
%\fonte{Universidade Federal de Juiz de Fora (2012).} 
%\nota{Ilustração incompleta.} %%Indicar a fonte consultada (elemento obrigat\'orio, mesmo que seja produ\c{c}\~ao do pr\'oprio autor).
%\end{center}
%\end{figure}


%%Caso a ilustracao seja elaborada pelo autor, usar ``\fonte{Elaborado pelo autor. (ano).}'' substituindo, se necessario, autor por autora ou Elaborado por Elaborada.

           %%Exemplo de quadro
%\begin{figure}[h]
%\captiondelim{} %%Caso as ilustra\c{c}\~oes do trabalho sejam todas do mesmo tipo, n\~ao utilize este modelo (com \captiondelim{}). Utilize o do final deste arquivo.
%\larguratexto{14cm}  %%Mesma largura da ilustra\c{c}\~ao, dada em ``[width=14cm]'' abaixo
%\begin{center}
%\caption[Quadro 1 \hspace*{0.1pt} -- Bibliotecas da UFJF %%\hspace*{...} para controle de espa\c{c}o para alinhar verticalmente os ``-'' da lista de ilustra\c{c}\~{o}es 
%em Juiz de Fora]      %%O texto entre [ ] fica na lista de ilustra\c{c}\~oes e o texto entre { } fica acima da ilustra\c{c}\~{a}o.
%{Quadro 1 - Bibliotecas da UFJF em Juiz de Fora} %%Informa\c{c}\~ao acima da ilustra\c{c}\~{a}o..
%\includegraphics[width=14cm]{bibliotecas.png}
%\fonte{Universidade Federal de Juiz de Fora (2012).} %%Indicar a fonte consultada (elemento obrigat\'orio, mesmo que seja produ\c{c}\~ao do pr\'oprio autor).
%\end{center}
%\end{figure}

%Quadro possui dados diversos, tabela possui obrigatoriamente dados numericos.

           %%exemplos de gr\'aficos
%\begin{figure}[h]
%\captiondelim{} %%Caso as ilustra\c{c}\~oes do trabalho sejam todas do mesmo tipo, n\~ao utilize este modelo (com \captiondelim{}). Utilize o do final deste arquivo.
%\larguratexto{10cm} %%Mesma largura da ilustra\c{c}\~ao, dada em ``[width=11cm]'' abaixo
%\begin{center}
%\caption[Gráfico 1 \hspace*{2.5pt} -- \'Indice de qualifica\c{c}\~{a}o do corpo docente da UFJF %%\hspace*{...} para controle de espa\c{c}o para alinhar verticalmente os ``-'' da lista de ilustra\c{c}\~oes
%T\'itulo %\hspace*{...} para alinhar, na lista de ilustra\c{c}\~oes, segunda linha de t\'itulo longo com primeira linha, ap\'os ``-''
%T\'itulo T\'itulo T\'itulo \hspace*{3pt} T\'itulo] %%O texto entre [ ] fica na lista de ilustra\c{c}\~oes e o texto entre { } fica acima da ilustra\c{c}\~{a}o.
%{Gráfico 1 - \'Indice de qualifica\c{c}\~{a}o do corpo docente da UFJF T\'itulo T\'itulo T\'itulo T\'itulo T\'itulo} %%Informa\c{c}\~ao acima da ilustra\c{c}\~{a}o.
%\includegraphics[width=10cm]{qualific.png} 
%\fonte{Universidade Federal de Juiz de Fora (2012).} %%Indicar a fonte consultada (elemento obrigat\'orio, mesmo que seja produ\c{c}\~ao do pr\'oprio autor).
%\end{center}
%\end{figure}           
           
%\begin{figure}[h!]
%\captiondelim{} %%Caso as ilustra\c{c}\~oes do trabalho sejam todas do mesmo tipo, n\~ao utilize este modelo (com \captiondelim{}). Utilize o do final deste arquivo.
%\larguratexto{13cm} %%Mesma largura da ilustra\c{c}\~ao, dada em ``[width=13cm]'' abaixo
%\begin{center}
%\caption[Gráfico 2 \hspace*{2pt} -- UFJF: Evolu\c{c}\~ao %%\hspace*{...} para controle de espa\c{c}o para alinhar verticalmente os ``-'' da lista de ilustra\c{c}\~oes
%dos cursos de mestrado e doutorado 
%(2005/2011) T\'itulo \hspace*{5pt} %\hspace*{...} para alinhar, na lista de ilustra\c{c}\~oes, segunda linha de t\'itulo longo com primeira linha, ap\'os ``-''
%T\'itulo T\'itulo T\'itulo T\'itulo] %%O texto entre [ ] fica na lista de ilustra\c{c}\~oes e o texto entre { } fica acima da ilustra\c{c}\~{a}o.
%{Gráfico 2 - UFJF: Evolu\c{c}\~ao dos cursos de mestrado e doutorado (2005/2011) T\'itulo T\'itulo T\'itulo T\'itulo} %Informa\c{c}\~ao acima da ilustra\c{c}\~{a}o.
%\includegraphics[width=13cm]{mest_dout.png} 
%\fonte{Universidade Federal de Juiz de Fora (2012).} %Indicar a fonte consultada (elemento obrigat\'orio, mesmo que seja produ\c{c}\~ao do pr\'oprio autor).
%\end{center}
%\end{figure}


\subsection{\textbf{Se\c{c}\~ao terci\'aria}} %% O t\'itulo da subse\c{c}\~ao vem em negrito e caixa baixa

Abaixo, s\~ao apresentados exemplos de tabela. 

%%Exemplo de tabela. Tabelas nao possuem margem lateral. Tabelas apresentam obrigatoriamente dados numericos.

%\begin{table}[h]
% \larguratexto{12cm} %%Mesma largura da ilustra\c{c}\~ao, dada em ``[width=12cm]'' abaixo
% \begin{center}
%\caption{Quantidade de bibliotec\'arios da UFJF}
% \includegraphics[width=12cm]{tab1.png}
% \fonte{Elaborada pelo autor (2019).} 
%\end{center}
%\end{table}

%\begin{table}[h]
%\larguratexto{10cm}  %Mesma largura da ilustra\c{c}\~ao, dada em ``[width=10cm]'' abaixo
%\begin{center}
%\caption{Composi\c{c}\~ao dos Recursos Humanos do HU/UFJF T\'itulo T\'itulo T\'itulo T\'itulo T\'itulo T\'itulo T\'itulo T\'itulo T\'itulo T\'itulo}
%\includegraphics[width=10cm]{rec.png}
%\fonte{Universidade Federal de Juiz de Fora (2012).} 
%\end{center}
%\end{table}

%%Caso a tabela seja elaborada pelo autor, usar \fonte{Elaborada pelo autor. (ano).} substituindo, se necessario, autor por autora.

\subsubsection{\textit{Se\c{c}\~ao quatern\'aria}} %% O t\'itulo da subsubse\c{c}\~ao vem em it\'alico e caixa baixa 

Se houver se\c{c}\~ao quatern\'aria, incluir texto ...

\subsubsubsection{Se\c{c}\~ao quin\'aria}  %% O t\'itulo desta vem em caixa baixa

Se houver se\c{c}\~ao quin\'aria, incluir texto ...


\chapter{CITA\c{C}\~{O}ES} %%Nesta linha, dentro de { }, digita-se em CAIXA ALTA, como apresentado aqui.

As citações são informa\c{c}\~{o}es extra\'idas de fonte consultada pelo autor da obra em desenvolvimento. Podem ser diretas, indiretas ou citação de citação. Para exemplos, consultar o apêncice D no Manual de Normalização de Trabalhos Acadêmicos disponível no \textit{link} abaixo: \\ 
\url{https://www2.ufjf.br/biblioteca/servicos/#normalizacao-bibliografica}

\section{SISTEMA AUTOR-DATA} %%Nesta linha, dentro de { }, digita-se o nome da se\c{c}\~ao secund\'aria em CAIXA ALTA, como apresentado aqui.

Para o sistema autor-data, considere: 
\begin{itemize}
 \item[a)] \textbf{citação direta} \'e caracterizada pela transcri\c{c}\~{a}o textual da parte consultada. Se com at\'e tr\^es linhas, deve estar entre aspas duplas, exatamente como na obra consultada. Se com mais de tr\^es linhas, recomenda-se o recuo de 4 cm da margem esquerda, com letra menor (um ponto), espaçamento simples, sem aspas. Sendo a chamada: (Autor, data e p\'agina) ou na senten\c{c}a Autor (data, p\'agina).
 \item[b)] \textbf{cita\c{c}\~{a}o indireta} \'e aquela em que o texto foi baseado na(s) obra(s) consultada(s). Em caso de mais de tr\^es fontes consultadas, a cita\c{c}\~{a}o deve seguir a ordem alfab\'etica. 
 \item[c)] \textbf{A cita\c{c}\~{a}o de cita\c{c}\~{a}o} \'e baseada em um texto em que n\~ao houve acesso ao original. 
\end{itemize} 


 
\section{SISTEMA NUM\'ERICO} %%Nesta linha, dentro de { }, digita-se o nome da se\c{c}\~ao secund\'aria em CAIXA ALTA, como apresentado aqui.

\textbf{Para o sistema num\'erico:} 

A indica\c{c}\~{a}o da fonte \'e feita por uma numera\c{c}\~{a}o \'unica e consecutiva respeitando a ordem que aparece no texto. Deve-se usar algarismos ar\'abicos remetendo \`a lista de refer\^encias. A indica\c{c}\~{a}o da numera\c{c}\~{a}o \'e apresentada entre par\^enteses no corpo do texto ou como expoente. N\~ao usar colchetes. O autor pode aparecer ou n\~ao no texto. Para separar diversos autores, utiliza-se v\'irgula. N\~{a}o utilizar nota de rodap\'{e} quando utilizar o sistema num\'{e}rico.
Observe os exemplos no Manual de Normaliza\c{c}\~{a}o de Trabalhos Acad\^emicos dispon\'ivel no \textit{link} abaixo: \\
\url{https://www2.ufjf.br/biblioteca/servicos/#normalizacao-bibliografica}

Em citação direta, o número da página (precedido por ``p.'') ou localizador, se houver, deve ser indicado após o número da fonte no texto, separado por vírgula e um espaço. O número do localizador, em publicações eletrônicas, deve ser precedido por sua respectiva abreviatura (local.). Exemplos: (1, p. 30), (7, local. 72), (4, Mt 6, 3-6, p. 1730), (6, v.3, p.583), (5, cap. V, art. 49, inc.I), (2, 9 min 41 s).

\section{NOTAS} %%Nesta linha, dentro de { }, digita-se o nome da se\c{c}\~ao secund\'aria em CAIXA ALTA, como apresentado aqui.

Notas de rodap\'e s\~ao observa\c{c}\~{o}es e/ou aditamentos que o autor precisa incluir no texto \footnote[2]{As notas devem ser alinhadas sendo que na segunda linha da mesma nota, a primeira letra deve estar abaixo da primeira letra da primeira palavra da linha superior, destacando assim o expoente.}. Para a numera\c{c}\~{a}o das notas deve-se utilizar algarismos ar\'abicos. As notas devem ser digitadas dentro das margens, ficando separadas do texto por um espa\c{c}o simples entre as linhas e por filete de 5 cm a partir da margem esquerda e em fonte menor (um ponto) do corpo do texto. As notas de rodap\'e s\'o podem ser usadas no sistema autor-data. Observe os exemplos no Manual de Normaliza\c{c}\~{a}o de Trabalhos Acad\^emicos dispon\'ivel no \textit{link} abaixo: \\
\url{https://www2.ufjf.br/biblioteca/servicos/#normalizacao-bibliografica}

%%%%%%%%%%%%%%%
%%%%%%%%%%%%%%%
%%EXEMPLO DE AL\'INEAS

%\begin{alineas}
% \item texto;    
% \item texto; 
% \item texto.
%\end{alineas}

%%Existe tamb\'em ``\begin{subalineas} \item ... \end{subalineas}'' que em cada linha fica sem recuo e coloca - no lugar das letras do alfabeto.  
%%%%%%%%%%%%%%%
%%%%%%%%%%%%%%%

\chapter{CONCLUS\~AO} %%Nesta linha, dentro de { }, digita-se em CAIXA ALTA, como apresentado aqui.

Este elemento \'e obrigat\'orio e \'e a parte final do texto.  Nele, s\~ao apresentadas as conclus\~oes identificadas a partir do desenvolvimento da pesquisa.

%Todo trabalho deve conter apenas um elemento conclusivo.

%%%%%%%%%%%%%%%%%%
%%%%%%%%%%%%%%%%%%
%% ELEMENTOS POS-TEXTUAIS

\postextual 


%% Fizemos a op\c{c}\~ao por colocar as refer\^encias diretamente no arquivo ``.tex'' por ser mais simples para quem se inicia na escrita de trabalhos acad\^emicos.
%% Referencias. LISTAR EXATAMENTE AS CITADAS NO TRABALHO.

%No elemento REFER\^ENCIAS, todas ``as refer\^encias devem ser ... alinhadas \`a margem esquerda do texto ... (ABNT, 2018). 


\begin{thebibliography}{99}


%%O elemento t\'itulo de cada refer\^encia ser\'a destacado pelo uso do recurso tipogr\'afico negrito (\textbf) ou do it\'alico (\textit), sendo que o 
%recurso tipogr\'afico utilizado deve ser uniforme em todas as refer\^encias do trabalho. Recomendamos o uso do negrito.

%%%1) Exemplos de refer\^encias no sistema num\'erico

%%exemplo de parte de obra em meio eletr\^onico
\bibitem{disp2019} S\~AO PAULO (Estado). Secretaria do Meio Ambiente. Tratados e organiza\c{c}\~oes ambientais em mat\'eria de meio ambiente. \textit{In}: S\~AO
PAULO (Estado). Secretaria do Meio Ambiente. \textbf{Entendendo o meio ambiente}. S\~ao Paulo: Secretaria do Meio Ambiente, 1999. v. 1. Disponível em: 
http://www.bdt.org.br/sma/entendendo/atual.htm. Acesso em: 8 mar. 1999.


%%exemplo de livro
\bibitem{Bauman99} BAUMAN, Zygmunt. \textbf{Globaliza\c{c}\~ao}: as consequ\^encias humanas. Rio de Janeiro: Jorge Zahar, 1999.


%%exemplo de artigo de publica\c{c}\~ao peri\'odica
\bibitem{vet18} DOREA, R. D.; COSTA, J. N.; BATITA, J. M.; FERREIRA, M. M.; MENEZES, R. V.; SOUZA, T. S. Reticuloperitonite traum\'atica associada \`a esplenite 
e hepatite em bovino: relato de caso. \textbf{Veterin\'aria e Zootecnia}, S\~ao Paulo, v. 18, n. 4, p. 199-202, 2011. Supl. 3.

%%exemplo de trabalho acad\^emico (tese, dissertac\{c}\~ao, etc.)

\bibitem{Aguiar2009} AGUIAR, Andr\'e Andrade de. \textbf{Avalia\c{c}\~ao da microbiota bucal em pacientes sob uso cr\^onico de penicilina e benzatina}. 2009. 
Tese (Doutorado em Cardiologia) - Faculdade de Medicina, Universidade de S\~ao Paulo, S\~ao Paulo, 2009.

%%%2) Exemplos de refer\^encia no sistema autor-data. Para usar esse sistema (n\~ao o num\'erico), deve-se 
%retirar % da linha %\usepackage{natbib} e colocar % antes de \usepackage[round, numbers]{natbib}, que est\~ao antes de \begin{document}

%% \bibitem[AGUIAR(2009)Aguiar]{t1} AGUIAR, Andr\'e Andrade de. \textbf{Avalia\c{c}\~ao da microbiota bucal em pacientes sob uso cr\^onico de penicilina e benzatina}. 
%2009. Tese (Doutorado em Cardiologia) - Faculdade de Medicina, Universidade de S\~ao Paulo, S\~ao Paulo, 2009.

%% \bibitem[BAUMAN(1999)Bauman]{Bauman99} BAUMAN, Zygmunt. \textbf{Globaliza\c{c}\~ao}: as consequ\^encias humanas. Rio de Janeiro: Jorge Zahar, 1999.

%% \bibitem[S\~AO PAULO(2019)S\~ao Paulo]{disp2019} S\~AO PAULO (Estado). Secretaria do Meio Ambiente. Tratados e organiza\c{c}\~oes ambientais em mat\'eria de meio ambiente. \textit{In}: S\~AO
%% PAULO (Estado). Secretaria do Meio Ambiente. \textbf{Entendendo o meio ambiente}. S\~ao Paulo: Secretaria do Meio Ambiente, 1999. v. 1. Disponível em: 
%% http://www.bdt.org.br/sma/entendendo/atual.htm. Acesso em: 8 mar. 1999.

\end{thebibliography}

%% Apendices e Anexos nao devem ser subdivididos: A1, A2, etc.

%% Apendices

\begin{apendices}

\chapter{\apendseq T\'itulo} 
%%Digita-se o titulo do apendice mantendo-se, antes, o comando \apendseq, como indicado.

Este elemento \'e opcional. Apresenta um texto ou documento elaborado pelo autor com o objetivo de complementar sua argumenta\c{c}\~ao, 
sem preju\'izo da unidade nuclear do trabalho.

\end{apendices}

%% Anexos

\begin{anexos}

\chapter{\anexoseq T\'itulo} 
%%Digita-se o titulo do anexo mantendo-se, antes, o comando \anexoseq, como indicado.

Este elemento \'e opcional. Apresenta um texto ou documento \textbf{n\~ao} elaborado pelo autor com o objetivo de complementar ou comprovar sua 
argumenta\c{c}\~ao. 

  
\end{anexos}


%%% ---
\end{document}

%%%%EXEMPLO QUANDO SE TEM TODAS AS ILUSTRA\c{C}\~OES DO MESMO TIPO. POR EXEMPLO, ORGANOGRAMA.

%No meio do texto acima:
%1) coloque % antes de cada dos comandos \ilustvaria e \listilustvaria ;
%2) acrescente os dois comandos abaixo 

\tipoilust{Organograma} %Preencha com o tipo de sua ilustra\c{c}\~ao (somente caso todas sejam do mesmo tipo). Por exemplo, Organograma.
\renewcommand{\listfigurename}{\textbf{LISTA DE ORGANOGRAMAS}} %Troque ORGANOGRAMAS por outra palavra conforme o tipo de sua ilustra\c{c}\~ao, se for \'unico.

%3) retire % do in\'icio do comando 
\listoffigures* %Use este comando quando todas as ilustra\c{c}\~oes s\~ao do mesmo tipo e caso queira inserir a lista delas.

%Exemplo para se colocar a ilustrac\{c}\~ao neste caso, de tipo \'unico (por exemplo, Organograma) em todo o trabalho.

\begin{figure}[h]
\larguratexto{6cm}  %Mesma largura da ilustra\c{c}\~ao, dada em ``[width=6cm]'' abaixo
\begin{center}
\caption{Texto} %Substituir ``Texto'' pela informa\c{c}\~ao acima da ilustra\c{c}\~{a}o.
\includegraphics[width=6cm]{arquivo.jpg}
\fonte{Universidade Federal de Juiz de Fora (2025).} %%Indicar a fonte consultada (elemento obrigat\'orio, mesmo que seja produ\c{c}\~ao do pr\'oprio autor).
\end{center}
\end{figure}

